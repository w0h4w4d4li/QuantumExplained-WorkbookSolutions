\documentclass{article}
\usepackage{amsmath}  % For advanced math typesetting
\usepackage{amsfonts} % For math fonts
\usepackage{amssymb}  % For additional symbols
\usepackage{graphicx} % For including images
\usepackage{geometry} % For page layout
\usepackage{fancyhdr} % For header and footer
\usepackage{setspace} % For line spacing
\usepackage{hyperref} % For hyperlinks
\usepackage{titlesec} % For customizing section titles
\usepackage{tikz} % For drawing
\usepackage{braket} % For braket notation
\usepackage{cleveref}

\geometry{a4paper, margin=1in}

% Header and Footer
\pagestyle{fancy}
\fancyhf{}
\fancyhead[L]{Solutions to MidTerm}
\fancyhead[R]{\thepage}
\fancyfoot[C]{\thepage}

% Title Formatting
\titleformat{\section}{\Large\bfseries}{\thesection}{1em}{}
\titleformat{\subsection}{\large\bfseries}{\thesubsection}{1em}{}

% Title Page
\title{\textbf{Solutions to MidTerm}}
\author{
    MohamadAli Khajeian\footnote{khajeian@ut.ac.ir} \\ 
    \small \textit{Faculty of Engineering Sciences, University of Tehran, Iran} \\ 
}
\date{\today}

% Commands
\newcommand{\op}[2]{|#1\rangle \langle#2|}
\newcommand{\sand}[3]{\braket{#1 | #2 | #3}}
\newcommand{\sandop}[3]{\braket{#1 #2 #3}}

\begin{document}

\maketitle

\section*{Q1 - Solution}
\begin{align*}
   \ket{\psi} = \frac{2}{\sqrt{5}}\ket{0} + \frac{1}{\sqrt{5}}\ket{1},\quad \ket{\phi} = \frac{1}{\sqrt{2}}\ket{0} + \frac{1}{\sqrt{2}}\ket{1}
\end{align*}
\subsection*{(A)}
the density operator for $\ket{\psi}$ is
\begin{align*}
   \rho_{\psi} &= \op{\psi}{\psi} = \big(\frac{2}{\sqrt{5}}\ket{0} + \frac{1}{\sqrt{5}}\ket{1}\big)\big(\frac{2}{\sqrt{5}}\bra{0} + \frac{1}{\sqrt{5}}\bra{1}\big) \\
   &= \frac{4}{5}\op{0}{0} + \frac{2}{5}\op{0}{1} + \frac{2}{5}\op{1}{0} + \frac{1}{5}\op{1}{1}
\end{align*}
to show it is pure state, we have
\begin{align*}
   \rho^2_{\psi} &= \big(\frac{4}{5}\op{0}{0} + \frac{2}{5}\op{0}{1} + \frac{2}{5}\op{1}{0} + \frac{1}{5}\op{1}{1}\big)\big(\frac{4}{5}\op{0}{0} + \frac{2}{5}\op{0}{1} + \frac{2}{5}\op{1}{0} + \frac{1}{5}\op{1}{1}\big) \\
   &= \frac{16}{25}\op{0}{0} + \frac{8}{25}\op{0}{1} + \frac{4}{25}\op{0}{0} + \frac{2}{25}\op{0}{1} + \frac{8}{25}\op{1}{0} + \frac{4}{25}\op{1}{1} + \frac{2}{25}\op{1}{0} + \frac{1}{25}\op{1}{1}
\end{align*}
then we can get trace 
\begin{align*}
   \text{Tr}(\rho^2_{\psi}) = \sum_{i=0}^{1} \sand{i}{\rho^2_{\psi}}{i} = \frac{16}{25} + \frac{4}{25} + \frac{4}{25} + \frac{1}{25} = 1
\end{align*}
to get probability the system finding in state $\ket{0}$
\begin{align*}
   \rho_{\psi}\text{P}_{0} &= \big(\frac{4}{5}\op{0}{0} + \frac{2}{5}\op{0}{1} + \frac{2}{5}\op{1}{0} + \frac{1}{5}\op{1}{1}\big)\big(\op{0}{0}\big) \\
   &= \frac{4}{5}\op{0}{0} + \frac{2}{5}\op{1}{0}
\end{align*}
then we can get trace 
\begin{align*}
   \text{Tr}(\rho_{\psi}\text{P}_{0}) = \sum_{i=0}^{1} \sand{i}{\rho_{\psi}\text{P}_{0}}{i} = \frac{4}{5}
\end{align*}
to get probability the system finding in state $\ket{1}$
\begin{align*}
   \rho_{\psi}\text{P}_{1} &= \big(\frac{4}{5}\op{0}{0} + \frac{2}{5}\op{0}{1} + \frac{2}{5}\op{1}{0} + \frac{1}{5}\op{1}{1}\big)\big(\op{1}{1}\big) \\
   &= \frac{2}{5}\op{0}{1} + \frac{1}{5}\op{1}{1}
\end{align*}
then we can get trace 
\begin{align*}
   \text{Tr}(\rho_{\psi}\text{P}_{1}) = \sum_{i=0}^{1} \sand{i}{\rho_{\psi}\text{P}_{1}}{i} = \frac{1}{5}
\end{align*}
and the density operator for $\ket{\phi}$ is
\begin{align*}
   \rho_{\phi} &= \op{\phi}{\phi} = \big(\frac{1}{\sqrt{2}}\ket{0} + \frac{1}{\sqrt{2}}\ket{1}\big)\big(\frac{1}{\sqrt{2}}\bra{0} + \frac{1}{\sqrt{2}}\bra{1}\big) \\
   &= \frac{1}{2}\op{0}{0} + \frac{1}{2}\op{0}{1} + \frac{1}{2}\op{1}{0} + \frac{1}{2}\op{1}{1}
\end{align*}
to show it is pure state, we have
\begin{align*}
   \rho^2_{\phi} &= \frac{1}{4}\big(\op{0}{0} + \op{0}{1} + \op{1}{0} + \op{1}{1}\big)\big(\op{0}{0} + \op{0}{1} + \op{1}{0} + \op{1}{1}\big) \\
   &= \frac{1}{4}\big( \op{0}{0} + \op{0}{1} + \op{0}{0} + \op{0}{1} + \op{1}{0} + \op{1}{1} + \op{1}{0} + \op{1}{1} \big)
\end{align*}
then we can get trace 
\begin{align*}
   \text{Tr}(\rho^2_{\phi}) = \sum_{i=0}^{1} \sand{i}{\rho^2_{\phi}}{i} = \frac{1}{4} + \frac{1}{4} + \frac{1}{4} + \frac{1}{4} = 1
\end{align*}
to get probability the system finding in state $\ket{0}$
\begin{align*}
   \rho_{\phi}\text{P}_{0} &= \big(\frac{1}{2}\op{0}{0} + \frac{1}{2}\op{0}{1} + \frac{1}{2}\op{1}{0} + \frac{1}{2}\op{1}{1}\big)\big(\op{0}{0}\big) \\
   &= \frac{1}{2}\op{0}{0} + \frac{1}{2}\op{1}{0}
\end{align*}
then we can get trace 
\begin{align*}
   \text{Tr}(\rho_{\phi}\text{P}_{0}) = \sum_{i=0}^{1} \sand{i}{\rho_{\phi}\text{P}_{0}}{i} = \frac{1}{2}
\end{align*}
to get probability the system finding in state $\ket{1}$
\begin{align*}
   \rho_{\phi}\text{P}_{1} &= \big(\frac{1}{2}\op{0}{0} + \frac{1}{2}\op{0}{1} + \frac{1}{2}\op{1}{0} + \frac{1}{2}\op{1}{1}\big)\big(\op{1}{1}\big) \\
   &= \frac{1}{2}\op{0}{1} + \frac{1}{2}\op{1}{1}
\end{align*}
then we can get trace 
\begin{align*}
   \text{Tr}(\rho_{\phi}\text{P}_{1}) = \sum_{i=0}^{1} \sand{i}{\rho_{\phi}\text{P}_{1}}{i} = \frac{1}{2}
\end{align*}
\subsection*{(B)}
we determine the density operator for the ensemble
\begin{align*}
   \rho &= \sum_{i=0}^{1} \hat{p}_{i} \rho_{i} =  \frac{1}{4}\op{\psi}{\psi} + \frac{3}{4}\op{\phi}{\phi} \\
   &= \frac{1}{4}\big(\frac{4}{5}\op{0}{0} + \frac{2}{5}\op{0}{1} + \frac{2}{5}\op{1}{0} + \frac{1}{5}\op{1}{1}\big) + \frac{3}{4}\big(\frac{1}{2}\op{0}{0} + \frac{1}{2}\op{0}{1} + \frac{1}{2}\op{1}{0} + \frac{1}{2}\op{1}{1}\big) \\
   &= (\frac{1}{5} + \frac{3}{8})\op{0}{0} + (\frac{2}{20} + \frac{3}{8})\op{0}{1} + (\frac{2}{20} + \frac{3}{8})\op{1}{0} + (\frac{1}{20} + \frac{3}{8})\op{1}{1} \\
   &= (\frac{23}{40})\op{0}{0} + (\frac{19}{40})\op{0}{1} + (\frac{19}{40})\op{1}{0} + (\frac{17}{40})\op{1}{1} 
\end{align*}
\subsection*{(C)}
now, can get the trace
\begin{align*}
   \text{Tr}(\rho) = \sum_{i=0}^{1} \sand{i}{\rho}{i} = \frac{23}{40} + \frac{17}{40} = 1
\end{align*}
\subsection*{(D)}
to get probability the system finding in state $\ket{0}$
\begin{align*}
   \rho\text{P}_{0} &= \big((\frac{23}{40})\op{0}{0} + (\frac{19}{40})\op{0}{1} + (\frac{19}{40})\op{1}{0} + (\frac{17}{40})\op{1}{1}\big)\big(\op{0}{0}\big) \\
   &= (\frac{23}{40})\op{0}{0} + (\frac{19}{40})\op{1}{0}
\end{align*}
then we can get trace 
\begin{align*}
   \text{Tr}(\rho\text{P}_{0}) = \sum_{i=0}^{1} \sand{i}{\rho\text{P}_{0}}{i} = \frac{23}{40}
\end{align*}
to get probability the system finding in state $\ket{1}$
\begin{align*}
   \rho\text{P}_{1} &= \big((\frac{23}{40})\op{0}{0} + (\frac{19}{40})\op{0}{1} + (\frac{19}{40})\op{1}{0} + (\frac{17}{40})\op{1}{1}\big)\big(\op{1}{1}\big) \\
   &= (\frac{19}{40})\op{0}{1} + (\frac{17}{40})\op{1}{1}
\end{align*}
then we can get trace 
\begin{align*}
   \text{Tr}(\rho\text{P}_{1}) = \sum_{i=0}^{1} \sand{i}{\rho\text{P}_{1}}{i} = \frac{17}{40}.
\end{align*}
\section*{Q2 - Solution}
\[
A = \begin{pmatrix} 2 & 1 \\ -1 & -1 \end{pmatrix}
\]

\[
\det(A - \lambda I) = \det \begin{pmatrix} 2 - \lambda & 1 \\ -1 & -1 - \lambda \end{pmatrix} = 0
\]

\[
(2 - \lambda)(-1 - \lambda) - (-1)(1) = \lambda^2 - \lambda - 1 = 0
\]


\[
\lambda_{1,2} = \frac{1 \pm \sqrt{5}}{2}
\]
\section*{Q3 - Solution}

Assume we have state vector $\ket{\psi}$ and we want to transform two orthonormal basis to each other $\{\ket{u_{i}}\} \rightleftarrows \{\ket{v_{j}}\}$

\begin{equation}
    \label{1}
    \ket{\psi}_{u} = \sum_{i} c_{i}\ket{u_{i}}, \quad c_{i} = \braket{u_{i}|\psi}
\end{equation}
\begin{equation}
    \label{2}
    \ket{\psi}_{v} = \sum_{j} d_{j}\ket{v_{j}}, \quad d_{j} = \braket{v_{j}|\psi}
\end{equation}
Let's start with $\{\ket{u_{i}}\} \rightarrow \{\ket{v_{j}}\}$
\begin{equation*}
    d_{j} = \braket{v_{j}|\psi} = \braket{v_{j}| \hat{\textnormal{I}} | \psi} = \braket{v_{j}| \big( \sum_{i} |u_{i}\rangle \langle u_{i}|\big) | \psi}
    = \sum_{i} \braket{v_{j}|u_{i}}\braket{u_{i}|\psi}
\end{equation*}
According to \ref{1} and $\braket{v_{j}|u_{i}} = S_{ji}$ 
\begin{equation}
    d_{j} = \sum_{i} \braket{v_{j}|u_{i}}c_{i} = \sum_{i} S_{ji}c_{i}
\end{equation}
Thus $S$ is our Similarity Matrix, so we can say
\begin{equation}
    \ket{\psi}_{v} = S\ket{\psi}_{u}
\end{equation}
We can repeat this for $\{\ket{v_{j}}\} \rightarrow \{\ket{u_{i}}\}$
\begin{equation*}
    c_{i} = \braket{u_{i}|\psi} = \braket{u_{i}| \hat{\textnormal{I}} | \psi} = \braket{u_{i}| \big( \sum_{j} |v_{j}\rangle \langle v_{j}|\big) | \psi}
    = \sum_{j} \braket{u_{i}|v_{j}}\braket{v_{j}|\psi}
\end{equation*}
According to \ref{2} and $\braket{u_{i}|v_{j}} = \braket{v_{j}|u_{i}}^{*} = S_{ji}^{*}$ 
\begin{equation}
    c_{i} = \sum_{j} \braket{u_{i}|v_{j}}d_{j} = \sum_{j} S_{ji}^{*}d_{j}
\end{equation}
So we have
\begin{equation}
    \ket{\psi}_{u} = S^{\dagger}\ket{\psi}_{v}
\end{equation}
\\
Now suppose we want to transform the matrix representation of an operator in one basis like $\hat{\textnormal{A}}^{u}$ to representation of that operator in another basis
like $\hat{\textnormal{A}}^{v}$
\begin{equation}
    \label{7}
    \hat{\textnormal{A}}^{u} = \sum_{i,j}\textnormal{A}_{ij}^{u}|u_{j}\rangle \langle u_{i}|, \quad \textnormal{A}_{ij}^{u} = \braket{u_{i}|\hat{\textnormal{A}}|u_{j}}
\end{equation}
\begin{equation}
    \label{8}
    \hat{\textnormal{A}}^{v} = \sum_{k,l}\textnormal{A}_{kl}^{v}|v_{k}\rangle \langle v_{l}|, \quad \textnormal{A}_{kl}^{v} = \braket{v_{k}|\hat{\textnormal{A}}|v_{l}}
\end{equation}
Let's start with $\textnormal{A}_{kl}^{v}$
\begin{equation*}
    \textnormal{A}_{kl}^{v} = \braket{v_{k}|\hat{\textnormal{A}}|v_{l}} = \braket{v_{k}|\hat{\textnormal{I}}\hat{\textnormal{A}}\hat{\textnormal{I}}|v_{l}}
    = \braket{v_{k}|\big(\sum_{i} |u_{i}\rangle \langle u_{i}| \big)\hat{\textnormal{A}}\big(\sum_{j} |u_{j}\rangle \langle u_{j}| \big)|v_{l}} \\
\end{equation*}
\begin{equation*}
    = \sum_{i,j} \braket{v_{k}|u_{i}} \braket{u_{i}| \hat{\textnormal{A}} | u_{j}} \braket{u_{j}| v_{l}}
\end{equation*}
According to \ref{7}, $\braket{v_{k}|u_{i}} = S_{ki}$ and $\braket{u_{j}| v_{l}} = S_{lj}^{*}$,
we can write
\begin{equation}
    \textnormal{A}_{kl}^{v} = \sum_{i,j} \braket{v_{k}|u_{i}} \braket{u_{i}| \textnormal{A} | u_{j}} \braket{u_{j}| v_{l}} = \sum_{i,j} S_{ki} \textnormal{A}_{ij}^{u} S_{lj}^{*}
\end{equation}
Thus $S$ is our Similarity Matrix, so we can say
\begin{equation}
    \hat{\textnormal{A}}^{v} = S\hat{\textnormal{A}}^{u}S^{\dagger}
\end{equation}
We can repeat this for $\textnormal{A}_{ij}^{u}$
\begin{equation*}
    \textnormal{A}_{ij}^{u} = \braket{u_{i}|\hat{\textnormal{A}}|u_{j}} = \braket{u_{i}|\hat{\textnormal{I}}\hat{\textnormal{A}}\hat{\textnormal{I}}|u_{j}}
    = \braket{u_{i}|\big(\sum_{k} |v_{k}\rangle \langle v_{k}| \big)\hat{\textnormal{A}}\big(\sum_{l} |v_{l}\rangle \langle v_{l}| \big)|u_{j}} \\
\end{equation*}
\begin{equation*}
    = \sum_{k,l} \braket{u_{i}|v_{k}} \braket{v_{k}| \hat{\textnormal{A}} | v_{l}} \braket{v_{l}| u_{j}}
\end{equation*}
According to \ref{8}, $\braket{u_{i}|v_{k}} = S_{ki}^{*}$ and $\braket{u_{l}| v_{j}} = S_{lj}$,
we can write
\begin{equation}
    \textnormal{A}_{ij}^{u} = \sum_{k,l} \braket{u_{i}|v_{k}} \braket{v_{k}| \hat{\textnormal{A}} | v_{l}} \braket{v_{l}| u_{j}} = \sum_{k,l} S_{ki}^{*} \textnormal{A}_{kl}^{v} S_{lj}
\end{equation}
Thus $S$ is our Similarity Matrix, so we can say
\begin{equation}
    \hat{\textnormal{A}}^{u} = S^{\dagger}\hat{\textnormal{A}}^{v}S
\end{equation}

\section*{Q4 - Solution}
\[
H = \frac{1}{2} \sum_{i=0}^3 (\sigma_i \otimes \sigma_i),
\]
where
\[
\sigma_0 = I = \begin{bmatrix} 1 & 0 \\ 0 & 1 \end{bmatrix}, \quad
\sigma_1 = \sigma_x = \begin{bmatrix} 0 & 1 \\ 1 & 0 \end{bmatrix}, \quad
\sigma_2 = \sigma_y = \begin{bmatrix} 0 & -i \\ i & 0 \end{bmatrix}, \quad
\sigma_3 = \sigma_z = \begin{bmatrix} 1 & 0 \\ 0 & -1 \end{bmatrix}.
\]
1. \( \sigma_0 \otimes \sigma_0 \):
\[
\sigma_0 \otimes \sigma_0 = 
\begin{bmatrix}
1 & 0 \\
0 & 1
\end{bmatrix} \otimes 
\begin{bmatrix}
1 & 0 \\
0 & 1
\end{bmatrix} =
\begin{bmatrix}
1 & 0 & 0 & 0 \\
0 & 1 & 0 & 0 \\
0 & 0 & 1 & 0 \\
0 & 0 & 0 & 1
\end{bmatrix}.
\]
2. \( \sigma_x \otimes \sigma_x \):
\[
\sigma_x \otimes \sigma_x =
\begin{bmatrix}
0 & 1 \\
1 & 0
\end{bmatrix} \otimes
\begin{bmatrix}
0 & 1 \\
1 & 0
\end{bmatrix} =
\begin{bmatrix}
0 & 0 & 0 & 1 \\
0 & 0 & 1 & 0 \\
0 & 1 & 0 & 0 \\
1 & 0 & 0 & 0
\end{bmatrix}.
\]
3. \( \sigma_y \otimes \sigma_y \):
\[
\sigma_y \otimes \sigma_y =
\begin{bmatrix}
0 & -i \\
i & 0
\end{bmatrix} \otimes
\begin{bmatrix}
0 & -i \\
i & 0
\end{bmatrix} =
\begin{bmatrix}
0 & 0 & 0 & -1 \\
0 & 0 & 1 & 0 \\
0 & 1 & 0 & 0 \\
-1 & 0 & 0 & 0
\end{bmatrix}.
\]
4. \( \sigma_z \otimes \sigma_z \):
\[
\sigma_z \otimes \sigma_z =
\begin{bmatrix}
1 & 0 \\
0 & -1
\end{bmatrix} \otimes
\begin{bmatrix}
1 & 0 \\
0 & -1
\end{bmatrix} =
\begin{bmatrix}
1 & 0 & 0 & 0 \\
0 & -1 & 0 & 0 \\
0 & 0 & -1 & 0 \\
0 & 0 & 0 & 1
\end{bmatrix}.
\]
Now combine all terms:
\[
H = \frac{1}{2} \left( \sigma_0 \otimes \sigma_0 + \sigma_x \otimes \sigma_x + \sigma_y \otimes \sigma_y + \sigma_z \otimes \sigma_z \right).
\]
Substituting:
\[
H = \frac{1}{2} \left(
\begin{bmatrix}
1 & 0 & 0 & 0 \\
0 & 1 & 0 & 0 \\
0 & 0 & 1 & 0 \\
0 & 0 & 0 & 1
\end{bmatrix}
+
\begin{bmatrix}
0 & 0 & 0 & 1 \\
0 & 0 & 1 & 0 \\
0 & 1 & 0 & 0 \\
1 & 0 & 0 & 0
\end{bmatrix}
+
\begin{bmatrix}
0 & 0 & 0 & -1 \\
0 & 0 & 1 & 0 \\
0 & 1 & 0 & 0 \\
-1 & 0 & 0 & 0
\end{bmatrix}
+
\begin{bmatrix}
1 & 0 & 0 & 0 \\
0 & -1 & 0 & 0 \\
0 & 0 & -1 & 0 \\
0 & 0 & 0 & 1
\end{bmatrix}
\right).
\]
Adding term by term:
\[
H = \frac{1}{2} \begin{bmatrix}
2 & 0 & 0 & 0 \\
0 & 0 & 2 & 0 \\
0 & 2 & 0 & 0 \\
0 & 0 & 0 & 2
\end{bmatrix}.
\]
Simplify:
\[
H = \begin{bmatrix}
1 & 0 & 0 & 0 \\
0 & 0 & 1 & 0 \\
0 & 1 & 0 & 0 \\
0 & 0 & 0 & 1
\end{bmatrix}.
\]

\section*{(a)}
From the matrix:
\[
H = \begin{bmatrix}
1 & 0 & 0 & 0 \\
0 & 0 & 1 & 0 \\
0 & 1 & 0 & 0 \\
0 & 0 & 0 & 1
\end{bmatrix}.
\]
Squaring \( H \):
\[
H^2 = \begin{bmatrix}
1 & 0 & 0 & 0 \\
0 & 0 & 1 & 0 \\
0 & 1 & 0 & 0 \\
0 & 0 & 0 & 1
\end{bmatrix}
\begin{bmatrix}
1 & 0 & 0 & 0 \\
0 & 0 & 1 & 0 \\
0 & 1 & 0 & 0 \\
0 & 0 & 0 & 1
\end{bmatrix}.
\]
Performing the matrix multiplication:
\[
H^2 = \begin{bmatrix}
1 & 0 & 0 & 0 \\
0 & 1 & 0  & 0 \\
0 & 0 & 1 & 0 \\
0 & 0 & 0 & 1
\end{bmatrix} = I.
\]

\section*{(b)}
\begin{align*}
   \exp(-i \theta U) &= \cos\theta I - i \sin\theta U 
\end{align*}
so we have
\begin{align*}
   \exp(-i\pi H/4) &= \cos\pi/4 I - i \sin\pi/4 H \\
   \exp(-i\pi H/2) &= \cos\pi/2 I - i \sin\pi/2 H
\end{align*}

\section*{(c)}
\begin{align*}
   H = \begin{bmatrix}
      1 & 0 & 0 & 0 \\
      0 & 0 & 1 & 0 \\
      0 & 1 & 0 & 0 \\
      0 & 0 & 0 & 1
      \end{bmatrix}
\end{align*}
\begin{align*}
   \det(H - \lambda I) = \det\bigg(\begin{bmatrix}
      1-\lambda & 0 & 0 & 0 \\
      0 & -\lambda & 1 & 0 \\
      0 & 1 & -\lambda & 0 \\
      0 & 0 & 0 & 1-\lambda
      \end{bmatrix}
      \bigg) &= (1-\lambda)\det\bigg(\begin{bmatrix}
         -\lambda & 1 & 0 \\
         1 & -\lambda & 0 \\
         0 & 0 & 1-\lambda
         \end{bmatrix}\bigg) \\
      &= (1-\lambda)\bigg((-\lambda)(\det\bigg(\begin{bmatrix}
         -\lambda & 0 \\
         0 & 1-\lambda
         \end{bmatrix}\bigg)\big) \\ &-(\det\bigg(\begin{bmatrix}
            1 & 0 \\
            0 & 1-\lambda
            \end{bmatrix}\bigg)\bigg) \\
         &= (1-\lambda)\bigg((-\lambda)(-\lambda)(1-\lambda) - (1-\lambda)\bigg) \\
         &= (\lambda^2 - 1)(1-\lambda)^2 = (1+\lambda)(1-\lambda)^3 = 0
\end{align*}
so we have
\begin{align*}
   \lambda_1 = -1, \quad
   \lambda_{2,3,4} = 1
\end{align*}
\section*{Q5 - Solution}
\[
C = \frac{1}{\sqrt{5}} \begin{pmatrix} 1 & a \\  2i & b \end{pmatrix}, \quad
C^\dagger = \frac{1}{\sqrt{5}} \begin{pmatrix} 1 & -2i \\ \overline{a} & \overline{b} \end{pmatrix}
\]
\[
C^\dagger C = \frac{1}{5} \begin{pmatrix} 1 & -2i \\ \overline{a} & \overline{b} \end{pmatrix} \begin{pmatrix} 1 & a \\ 2i & b \end{pmatrix}.
\]
Perform the matrix multiplication:
\[
C^\dagger C = \frac{1}{5} \begin{pmatrix}
1 + 4 & a - 2ib \\
\overline{a} + 2i\overline{b} & |a|^2 + |b|^2
\end{pmatrix}.
\]
For \( C \) to be unitary, \( C^\dagger C = I \), which implies:
\[
\begin{aligned}
&1 + 4 = 5, \\
&a - 2ib = 0, \\
&|a|^2 + |b|^2 = 5.
\end{aligned}
\]
\begin{itemize}
\item From \( a - 2ib = 0 \), we get \( a = 2ib \).
\item Substitute \( a = 2ib \) into \( |a|^2 + |b|^2 = 5 \):
\[
|2ib|^2 + |b|^2 = 5.
\]
Since \( |2ib|^2 = 4|b|^2 \), we have:
\[
4|b|^2 + |b|^2 = 5 \quad \Rightarrow \quad 5|b|^2 = 5 \quad \Rightarrow \quad |b|^2 = 1.
\]
\item Thus, \( |b| = 1 \). Let \( b = e^{i\theta} \), where \( \theta \in \mathbb{R} \). Then \( a = 2i b = 2i e^{i\theta} \).
\end{itemize}
The possible pairs of \( a \) and \( b \) are:
\[
a = 2i e^{i\theta}, \quad b = e^{i\theta},
\]
\section*{Q6 - Solution}

\section*{(a)}

\[
\ket{0} \otimes H^{\otimes 3}(\ket{011}) \otimes \ket{1}
\]
Apply \( H^{\otimes 3} \) to \( \ket{011} \):
\[
H^{\otimes 3}(\ket{011}) = \frac{1}{\sqrt{8}} \sum_{a,b,c \in \{0,1\}} (-1)^{b+c} \ket{abc}
\]
then
\[
\ket{0} \otimes H^{\otimes 3}(\ket{011}) \otimes \ket{1} = \frac{1}{\sqrt{8}} \sum_{a,b,c \in \{0,1\}} (-1)^{b+c} \ket{0} \ket{abc} \ket{1}
\]
\[
\frac{1}{\sqrt{8}} \Big[
\ket{0}\ket{000}\ket{1} + \ket{0}\ket{001}\ket{1} - \ket{0}\ket{010}\ket{1} - \ket{0}\ket{011}\ket{1} 
\]
\[
- \ket{0}\ket{100}\ket{1} - \ket{0}\ket{101}\ket{1} + \ket{0}\ket{110}\ket{1} + \ket{0}\ket{111}\ket{1}
\Big]
\]

\section*{(b)}

\[
H^{\otimes 5}(\ket{+-++-})
\]
Apply \( H^{\otimes 5} \) to \( \ket{+-++-} \):
\begin{align*}
   H^{\otimes 5}(\ket{+-++-}) &= \big(H\ket{+} \otimes H\ket{-} \otimes H\ket{+} \otimes H\ket{+} \otimes H\ket{-}\big) \\
   &= \big(\ket{0} \otimes \ket{1} \otimes \ket{0} \otimes \ket{0} \otimes \ket{1}\big) = \ket{01001}
\end{align*}

\section*{Q7 - Solution}
\section*{(a)}
Consider the two-qubit state:
\[
\ket{\Phi}_{AB} = \frac{1}{\sqrt{2}} \bigg[ \ket{0}_A \bigg(\frac{1}{2}\ket{0}_B + \frac{\sqrt{3}}{2}\ket{1}_B \bigg) + \ket{1}_A \bigg(\frac{\sqrt{3}}{2}\ket{0}_B + \frac{1}{2}\ket{1}_B \bigg) \bigg].
\]
The density matrix of the total state is:
\[
\rho_{AB} = \ket{\Phi}_{AB} \bra{\Phi}_{AB}.
\]
Expand the state explicitly:
\[
\ket{\Phi}_{AB} = \frac{1}{\sqrt{2}} \bigg[ \frac{1}{2}\ket{0}_A \ket{0}_B + \frac{\sqrt{3}}{2}\ket{0}_A \ket{1}_B + \frac{\sqrt{3}}{2}\ket{1}_A \ket{0}_B + \frac{1}{2}\ket{1}_A \ket{1}_B \bigg].
\]
Let the coefficients \(c_{ij}\) for \(\ket{i}_A \ket{j}_B\) be:
\[
c_{00} = \frac{1}{2}, \quad c_{01} = \frac{\sqrt{3}}{2}, \quad c_{10} = \frac{\sqrt{3}}{2}, \quad c_{11} = \frac{1}{2}.
\]
The density matrix is:
\[
\rho_{AB} = \ket{\Phi}_{AB} \bra{\Phi}_{AB}.
\]
Writing this explicitly:
\[
\rho_{AB} = \frac{1}{2} \sum_{i,j,k,l} c_{ij} c_{kl}^* \ket{i}_A\ket{j}_B \bra{k}_A \bra{l}_B.
\]
Substituting the coefficients \(c_{ij}\), we can write:
\[
\rho_{AB} = \frac{1}{2} \bigg[ \frac{1}{4} \ket{00}\bra{00} + \frac{\sqrt{3}}{4} \ket{00}\bra{01} + \frac{\sqrt{3}}{4} \ket{00}\bra{10} + \frac{3}{4} \ket{00}\bra{11} + \cdots \bigg].
\]
The partial trace over \(B\) sums over the basis states \(\ket{j}_B\):
\[
\rho_A = \text{Tr}_B(\rho_{AB}) = \sum_{j=0}^1 \bra{j}_B \rho_{AB} \ket{j}_B.
\]
Extract terms corresponding to each \(\ket{j}_B\):
- For \(j = 0\), collect all terms where \(B\) is in \(\ket{0}_B\):
\[
\bra{0}_B \rho_{AB} \ket{0}_B = \frac{1}{2} \bigg[ \frac{1}{4} \ket{0}_A\bra{0} + \frac{\sqrt{3}}{4} \ket{1}_A\bra{0} + \cdots \bigg].
\]
- Similarly, for \(j = 1\), collect terms for \(\ket{1}_B\).
After computation, the reduced density matrix for \(A\) is:
\[
\rho_A = \begin{bmatrix}
\frac{5}{8} & \frac{\sqrt{3}}{8} \\
\frac{\sqrt{3}}{8} & \frac{3}{8}
\end{bmatrix}.
\]
The partial trace over \(A\) sums over the basis states \(\ket{i}_A\):
\[
\rho_B = \text{Tr}_A(\rho_{AB}) = \sum_{i=0}^1 \bra{i}_A \rho_{AB} \ket{i}_A.
\]
Following a similar procedure as for \(\rho_A\), we find:
\[
\rho_B = \begin{bmatrix}
\frac{5}{8} & \frac{\sqrt{3}}{8} \\
\frac{\sqrt{3}}{8} & \frac{3}{8}
\end{bmatrix}.
\]

\subsection*{(b)}

Both \(\rho_A\) and \(\rho_B\) have the same form:
\[
\rho = \begin{bmatrix}
\frac{5}{8} & \frac{\sqrt{3}}{8} \\
\frac{\sqrt{3}}{8} & \frac{3}{8}
\end{bmatrix}.
\]

The eigenvalues \(\lambda\) solve \(\det(\rho - \lambda I) = 0\):
\[
\lambda = \frac{1}{2}, \quad \lambda = 1.
\]

The eigenvectors are computed from \((\rho - \lambda I)\ket{v} = 0\):
\[
\ket{v_1} = \frac{1}{\sqrt{2}} \begin{bmatrix} 1 \\ 1 \end{bmatrix}, \quad \ket{v_2} = \frac{1}{\sqrt{2}} \begin{bmatrix} -1 \\ 1 \end{bmatrix}.
\]

Thus, both \(\rho_A\) and \(\rho_B\) are diagonalized as:
\[
\rho = \frac{1}{2} \ket{v_1}\bra{v_1} + \frac{1}{2} \ket{v_2}\bra{v_2}.
\]

\end{document}
