\documentclass[12pt]{article}
\usepackage{amsmath}  % For advanced math typesetting
\usepackage{amsfonts} % For math fonts
\usepackage{amssymb}  % For additional symbols
\usepackage{graphicx} % For including images
\usepackage{geometry} % For page layout
\usepackage{fancyhdr} % For header and footer
\usepackage{setspace} % For line spacing
\usepackage{hyperref} % For hyperlinks
\usepackage{titlesec} % For customizing section titles
\usepackage{tikz} % For drawing
\usepackage{braket} % For braket notation

\geometry{a4paper, margin=1in}

% Header and Footer
\pagestyle{fancy}
\fancyhf{}
\fancyhead[L]{Proof of relation (3.76)}
\fancyhead[R]{\thepage}
\fancyfoot[C]{\thepage}

% Title Formatting
\titleformat{\section}{\Large\bfseries}{\thesection}{1em}{}
\titleformat{\subsection}{\large\bfseries}{\thesubsection}{1em}{}

% Title Page
\title{\textbf{Proof of relation (3.76)}}
\author{
    MohamadAli Khajeian\footnote{khajeian@ut.ac.ir} \\ 
    \small \textit{Faculty of Engineering Sciences, University of Tehran, Iran} \\ 
}
\date{\today}

\begin{document}

\maketitle

\begin{abstract}
    This document presents the proof of relation (3.76) MacMahon book.
\end{abstract}


\section*{Relation (3.76)}

if $[A,B] \neq 0$ but $A$ and $B$ each commute with $[A,B]$. In that case

\begin{equation}
    \label{1}
    e^Ae^B=e^{A+B}e^{\frac{1}{2}[A,B]}.
\end{equation}

\section*{Proof}
Suppose that the commutator of two operators $A$, $B$
\begin{equation}
    [A,B]=c
\end{equation}
where $c$ commutes with $A$ and $B$, then
\begin{align}
    [A,e^{\lambda B}] &= \left[A,1+\lambda B+ \left(\dfrac{\lambda^2}{2!} \right)B^2+ \left(\dfrac{\lambda^3}{3!}\right)B^3+\dots\right] \\[5pt] &= \lambda c+ \left(\dfrac{\lambda^2}{2!}\right)2Bc+ \left(\dfrac{\lambda^3}{3!}\right)3B^2c+\dots \\[5pt] &= \lambda ce^{\lambda B}.
\end{align}
we can write
\begin{equation*}
    [A,e^{\lambda B}] = Ae^{\lambda B} - e^{\lambda B}A
\end{equation*}
\begin{equation*}
     Ae^{\lambda B} = e^{\lambda B}A + [A,e^{\lambda B}] 
\end{equation*}
so, we have
\begin{align} 
    e^{-\lambda B}Ae^{\lambda B} &= A+\lambda[A,B] = A+\lambda c. 
\end{align}
Consider $f(x)=e^{Ax}e^{Bx}$,
\begin{align*} 
    \frac{df}{dx} &=Ae^{Ax}e^{Bx}+e^{Ax}e^{Bx}B \\[4pt] &=f(x)(e^{-Bx}Ae^{Bx}+B) \\[4pt] &=f(x)(A+x[A,B]+B). 
\end{align*}
Let's solve this first-order differential equation,
\begin{align*}
    \frac{df}{dx} &= f(x)(A+x[A,B]+B) \\
    \frac{1}{f(x)}df &= (A+x[A,B]+B)dx \\
    \int \frac{1}{f(x)}df &= \int (A+x[A,B]+B)dx \\
    \ln{|f(x)|} &= (A+B)x + \frac{1}{2}x^{2}[A,B] + C \\
    f(x) &= e^{x(A+B)}e^{\frac{1}{2}x^2[A,B] + C} \\
\end{align*}
since $f(0) = I = e^{C}$ we get $C=0$,
\begin{align}
    f(x)=e^{x(A+B)}e^{\frac{1}{2}x^2[A,B]} \nonumber
\end{align}
so taking $x=1$ gives
\begin{equation*}
    e^Ae^B=e^{A+B}e^{\frac{1}{2}[A,B]}.
\end{equation*}
\end{document}
