\documentclass{article}
\usepackage{amsmath}  % For advanced math typesetting
\usepackage{amsfonts} % For math fonts
\usepackage{amssymb}  % For additional symbols
\usepackage{graphicx} % For including images
\usepackage{geometry} % For page layout
\usepackage{fancyhdr} % For header and footer
\usepackage{setspace} % For line spacing
\usepackage{hyperref} % For hyperlinks
\usepackage{titlesec} % For customizing section titles
\usepackage{tikz} % For drawing
\usepackage{braket} % For braket notation
\usepackage{cleveref}

\geometry{a4paper, margin=1in}

% Header and Footer
\pagestyle{fancy}
\fancyhf{}
\fancyhead[L]{Chapter 8 - Solutions to Try it}
\fancyhead[R]{\thepage}
\fancyfoot[C]{\thepage}

% Title Formatting
\titleformat{\section}{\Large\bfseries}{\thesection}{1em}{}
\titleformat{\subsection}{\large\bfseries}{\thesubsection}{1em}{}

% Title Page
\title{\textbf{Chapter 8} \\ \small Solutions to Try it}
\author{
    MohamadAli Khajeian\footnote{khajeian@ut.ac.ir} \\ 
    \small \textit{Faculty of Engineering Sciences, University of Tehran, Iran} \\ 
}
\date{\today}

% Commands
\newcommand{\op}[2]{|#1\rangle \langle#2|}
\newcommand{\sand}[3]{\braket{#1 | #2 | #3}}
\newcommand{\sandop}[3]{\braket{#1 #2 #3}}

\begin{document}

\maketitle

\begin{abstract}
    This document presents the solution of "Quantum Computing Explained by David McMAHON" exercises.
\end{abstract}

\section*{Try it - (page 175)}
Using De Morgan's laws, the OR operation is constructed as follows:
\[
\text{OR}(A, B) = \neg(\neg A \land \neg B) = \text{NAND}(\text{NAND}(A, A), \text{NAND}(B, B))
\]
\section*{Try it - (page 179)}
\begin{align*}
   U^{H}_{\text{NOT}} = 
   \begin{pmatrix}
      1 & 0 \\
      0 & -1 
   \end{pmatrix} = 
   \op{0}{0} - \op{1}{1}
\end{align*}
\begin{align*}
   \ket{+} &= \frac{1}{\sqrt{2}}\big(\ket{0}+\ket{1}\big) \\
   \ket{-} &= \frac{1}{\sqrt{2}}\big(\ket{0}-\ket{1}\big)
\end{align*}
now we can apply it on $\ket{+}$
\begin{align*}
   U^{H}_{\text{NOT}} \ket{+} = \big(\op{0}{0} - \op{1}{1}\big)\frac{1}{\sqrt{2}}\big(\ket{0}+\ket{1}\big) = \frac{1}{\sqrt{2}}\big(\ket{0}-\ket{1}\big) = \ket{-}
\end{align*}
also for $\ket{-}$
\begin{align*}
   U^{H}_{\text{NOT}} \ket{-} = \big(\op{0}{0} - \op{1}{1}\big)\frac{1}{\sqrt{2}}\big(\ket{0}-\ket{1}\big) = \frac{1}{\sqrt{2}}\big(\ket{0}+\ket{1}\big) = \ket{+}
\end{align*}
\section*{Try it - (page 181)}
\begin{align*}
   Y = -i\op{0}{1} + i\op{1}{0}
\end{align*}
\begin{align*}
   \ket{\psi} = \alpha\ket{0} + \beta\ket{1}
\end{align*}
we can apply $Y$ operator on it
\begin{align*}
   Y\ket{\psi} = \big(-i\op{0}{1} + i\op{1}{0}\big)\big(\alpha\ket{0} + \beta\ket{1}\big) = -i\beta\ket{0} + i\alpha\ket{1}
\end{align*}
\section*{Try it - (page 183)}
Hadamard operator is 
\begin{align*}
   H = \frac{1}{\sqrt{2}}\big(\op{0}{0}+\op{0}{1}+\op{1}{0}-\op{1}{1}\big)
\end{align*}
and our qubit is
\begin{align*}
   \ket{\psi} = \cos\theta\ket{0} + e^{i\phi}\sin\theta\ket{1}
\end{align*}
we can apply Hadamard on it
\begin{align*}
   H\ket{\psi} = \cos\theta H\ket{0} + e^{i\phi}\sin\theta H\ket{1} &= \cos\theta \ket{+} + e^{i\phi}\sin\theta \ket{-} \\
   &= \cos\theta \big(\frac{\ket{0}+\ket{1}}{\sqrt{2}}\big) + e^{i\phi}\sin\theta \big(\frac{\ket{0}-\ket{1}}{\sqrt{2}}\big) \\
   &= \big(\frac{\cos\theta\ket{0}+e^{i\phi}\sin\theta\ket{0}}{\sqrt{2}}\big) + \big(\frac{\cos\theta\ket{1}-e^{i\phi}\sin\theta\ket{1}}{\sqrt{2}}\big) \\
   &= \big(\frac{\cos\theta+e^{i\phi}\sin\theta}{\sqrt{2}}\big)\ket{0} +  \big(\frac{\cos\theta-e^{i\phi}\sin\theta}{\sqrt{2}}\big)\ket{1} = \ket{\psi^{'}}
\end{align*}
the probability that this measurement finds the system in the state $\ket{1}$
\begin{align*}
   \sand{\psi^{'}}{\text{P}_1}{\psi^{'}} &= \bigg(\frac{\cos\theta-e^{i\phi}\sin\theta}{\sqrt{2}}\bigg)^{*}\bigg(\frac{\cos\theta-e^{i\phi}\sin\theta}{\sqrt{2}}\bigg) \\
   &= \frac{1}{2}\big(\cos\theta-e^{-i\phi}\sin\theta\big)\big(\cos\theta-e^{i\phi}\sin\theta\big) \\
   &= \frac{1}{2}\big(\cos^2\theta + \sin^2\theta \big) - \frac{1}{2}\sin\theta\cos\theta\big(e^{i\phi}+e^{-i\phi}\big) = \frac{1}{2}\big(1 - \sin\theta\cos\theta\cos\phi\big)
\end{align*}
the probability that this measurement finds the system in the state $\ket{0}$
\begin{align*}
   \sand{\psi^{'}}{\text{P}_0}{\psi^{'}} &= \bigg(\frac{\cos\theta+e^{i\phi}\sin\theta}{\sqrt{2}}\bigg)^{*}\bigg(\frac{\cos\theta+e^{i\phi}\sin\theta}{\sqrt{2}}\bigg) \\
   &= \frac{1}{2}\big(\cos\theta+e^{-i\phi}\sin\theta\big)\big(\cos\theta+e^{i\phi}\sin\theta\big) \\
   &= \frac{1}{2}\big(\cos^2\theta + \sin^2\theta \big) + \frac{1}{2}\sin\theta\cos\theta\big(e^{i\phi}+e^{-i\phi}\big) = \frac{1}{2}\big(1 + \sin\theta\cos\theta\cos\phi\big)
\end{align*}
\section*{Try it - (page 184)}
\begin{align*}
   \exp(-i\theta U) = \cos\theta I - i\sin\theta U
\end{align*}
we can repalce $\theta=\displaystyle\frac{\gamma}{2}$ and $U = Z$,
\begin{align*}
   \exp(-i\frac{\gamma}{2} Z) = \cos\frac{\gamma}{2} I - i\sin\frac{\gamma}{2} Z &= \cos\frac{\gamma}{2}\big(\op{0}{0}+\op{1}{1}\big) - i\sin\frac{\gamma}{2}\big(\op{0}{0}-\op{1}{1}\big) \\
   &= \big(\cos\frac{\gamma}{2} - i\sin\frac{\gamma}{2}\big)\op{0}{0} + \big(\cos\frac{\gamma}{2} + i\sin\frac{\gamma}{2}\big)\op{1}{1} \\
   &= e^{-i\gamma/2}\op{0}{0} + e^{i\gamma/2}\op{1}{1} = R_{z}(\gamma)
\end{align*}
\end{document}
