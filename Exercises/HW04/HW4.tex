\documentclass[12pt]{article}
\usepackage{amsmath}  % For advanced math typesetting
\usepackage{amsfonts} % For math fonts
\usepackage{amssymb}  % For additional symbols
\usepackage{graphicx} % For including images
\usepackage{geometry} % For page layout
\usepackage{fancyhdr} % For header and footer
\usepackage{setspace} % For line spacing
\usepackage{hyperref} % For hyperlinks
\usepackage{titlesec} % For customizing section titles
\usepackage{tikz} % For drawing
\usepackage{braket} % For braket notation

\geometry{a4paper, margin=1in}

% Header and Footer
\pagestyle{fancy}
\fancyhf{}
\fancyhead[L]{EigenVectors and EigenValues of Pauli Operators}
\fancyhead[R]{\thepage}
\fancyfoot[C]{\thepage}

% Title Formatting
\titleformat{\section}{\Large\bfseries}{\thesection}{1em}{}
\titleformat{\subsection}{\large\bfseries}{\thesubsection}{1em}{}

% Title Page
\title{\textbf{EigenVectors and EigenValues of Pauli Operators}}
\author{
    MohamadAli Khajeian\footnote{khajeian@ut.ac.ir} \\ 
    \small \textit{Faculty of Engineering Sciences, University of Tehran, Iran} \\ 
}
\date{\today}

\begin{document}

\maketitle

\begin{abstract}
    This document presents the EigenVectors and EigenValues of Pauli Operators.
\end{abstract}

\section{The Pauli Operators}

\begin{equation}
\label{eq:1}
\sigma_{\textnormal{X}} = \textnormal{X} = 1\ket{0}\bra{1} + 1\ket{1}\bra{0}
\end{equation}
\begin{equation}
\label{eq:2}
\sigma_{\textnormal{Y}} = \textnormal{Y} = -i\ket{0}\bra{1} + i\ket{1}\bra{0}
\end{equation}
\begin{equation}
\label{eq:3}
\sigma_{\textnormal{Z}} = \textnormal{Z} = 1\ket{0}\bra{0} -1\ket{1}\bra{1}
\end{equation}

\section{EigenValues and EigenVectors}
According to
\begin{equation}
    \label{eq:ref}
    \hat{\textnormal{A}}\ket{\gamma} = \lambda\ket{\gamma}
\end{equation}
We assume $\ket{\gamma} = \alpha\ket{0}+\beta\ket{1}$.
\subsection{X}
Since \ref{eq:1}, we can find eigenvalues through
\begin{equation}
\det{(\sigma_{\textnormal{X}} - \lambda\textnormal{I})} = 0
\end{equation}
\begin{equation*}
\det{\big((-\lambda)\ket{0}\bra{0} + \ket{0}\bra{1} + \ket{1}\bra{0} + (-\lambda)\ket{1}\bra{1}\big)} = 0
\end{equation*}
\begin{equation*}
    \lambda^2 - 1 = 0 \Longrightarrow \lambda = \pm 1
\end{equation*}
For $\lambda_1 = 1$, from \ref{eq:ref} and \ref{eq:1} we have
\begin{equation}
    \label{eq:12}
    \sigma_{\textnormal{X}}\ket{\gamma} = \ket{\gamma} \Longrightarrow (1\ket{0}\bra{1} + 1\ket{1}\bra{0})(\alpha\ket{0}+\beta\ket{1}) = \beta\ket{0}+\alpha\ket{1} 
\end{equation}
From \ref{eq:12} we have
\begin{equation}
    \alpha = \beta
\end{equation}
To find $\alpha$ and $\beta$
\begin{equation}
    \|\alpha\|^2 + \|\beta\|^2 = 1 \Longrightarrow 2\|\alpha\|^2 = 1 \Longrightarrow \alpha = \beta = \frac{1}{\sqrt{2}}
\end{equation}
So, $\ket{\gamma_{1}}$ is
\begin{equation}
    \ket{\gamma_{1}} = \frac{1}{\sqrt{2}}(\ket{0} + \ket{1})
\end{equation}
For $\lambda_2 = -1$, from \ref{eq:ref} and \ref{eq:1} we have
\begin{equation}
    \label{eq:13}
    \sigma_{\textnormal{X}}\ket{\gamma} = -\ket{\gamma} \Longrightarrow (1\ket{0}\bra{1} + 1\ket{1}\bra{0})(\alpha\ket{0}+\beta\ket{1}) = -\beta\ket{0}-\alpha\ket{1} 
\end{equation}
From \ref{eq:13} we have
\begin{equation}
    \alpha = -\beta
\end{equation}
To find $\alpha$ and $\beta$
\begin{equation}
    \|\alpha\|^2 + \|\beta\|^2 = 1 \Longrightarrow 2\|\alpha\|^2 = 1 \Longrightarrow \alpha = - \beta = \frac{1}{\sqrt{2}}
\end{equation}
So, $\ket{\gamma_{2}}$ is
\begin{equation}
    \ket{\gamma_{2}} = \frac{1}{\sqrt{2}}(\ket{0} - \ket{1})
\end{equation}

\subsection{Y}
Since \ref{eq:2}, we can find eigenvalues through
\begin{equation}
\det{(\sigma_{\textnormal{Y}} - \lambda\textnormal{I})} = 0
\end{equation}
\begin{equation*}
\det{\big((-\lambda)\ket{0}\bra{0} + (-i)\ket{0}\bra{1} + i\ket{1}\bra{0} + (-\lambda)\ket{1}\bra{1}\big)} = 0
\end{equation*}
\begin{equation*}
    \lambda^2 - 1 = 0 \Longrightarrow \lambda = \pm 1
\end{equation*}
For $\lambda_1 = 1$, from \ref{eq:ref} and \ref{eq:2} we have
\begin{equation}
    \label{eq:21}
    \sigma_{\textnormal{Y}}\ket{\gamma} = \ket{\gamma} \Longrightarrow (-i\ket{0}\bra{1} + i\ket{1}\bra{0})(\alpha\ket{0}+\beta\ket{1}) = i(-\beta\ket{0}+\alpha\ket{1}) 
\end{equation}
From \ref{eq:21} we have
\begin{equation}
    \alpha = -i\beta, \quad \beta = i\alpha
\end{equation}
To find $\alpha$ and $\beta$
\begin{equation}
    \|\alpha\|^2 + \|\beta\|^2 = \|\alpha\|^2 + (-i\alpha^{*})(i\alpha) = 2\|\alpha\|^2 = 1 \Longrightarrow \alpha = \frac{1}{\sqrt{2}}
\end{equation}
So, $\ket{\gamma_{1}}$ is
\begin{equation}
    \ket{\gamma_{1}} = \frac{1}{\sqrt{2}}(\ket{0} + i\ket{1})
\end{equation}
For $\lambda_2 = -1$, from \ref{eq:ref} and \ref{eq:2} we have
\begin{equation}
    \label{eq:22}
    \sigma_{\textnormal{Y}}\ket{\gamma} = \ket{\gamma} \Longrightarrow (-i\ket{0}\bra{1} + i\ket{1}\bra{0})(\alpha\ket{0}+\beta\ket{1}) = i(-\beta\ket{0}+\alpha\ket{1}) 
\end{equation}
From \ref{eq:22} we have
\begin{equation}
    \alpha = i\beta, \quad \beta = -i\alpha
\end{equation}
To find $\alpha$ and $\beta$
\begin{equation}
    \|\alpha\|^2 + \|\beta\|^2 = \|\alpha\|^2 + (-i\alpha^{*})(i\alpha) = 2\|\alpha\|^2 = 1 \Longrightarrow \alpha = \frac{1}{\sqrt{2}}
\end{equation}
So, $\ket{\gamma_{2}}$ is
\begin{equation}
    \ket{\gamma_{2}} = \frac{1}{\sqrt{2}}(\ket{0} - i\ket{1})
\end{equation}

\subsection{Z}
Since \ref{eq:3}, we can find eigenvalues through
\begin{equation}
\det{(\sigma_{\textnormal{Z}} - \lambda\textnormal{I})} = 0
\end{equation}
\begin{equation*}
\det{\big((1-\lambda)\ket{0}\bra{0} + (-1-\lambda)\ket{1}\bra{1}\big)} = 0
\end{equation*}
\begin{equation*}
    (1-\lambda)(-1-\lambda) = -1 - \lambda + \lambda + \lambda^2 = \lambda^2 -1 = 0 \Longrightarrow \lambda = \pm 1
\end{equation*}
For $\lambda_1 = 1$, from \ref{eq:ref} and \ref{eq:3} we have
\begin{equation}
    \label{eq:24}
    \sigma_{\textnormal{Z}}\ket{\gamma} = \ket{\gamma} \Longrightarrow (\ket{0}\bra{0} - \ket{1}\bra{1})(\alpha\ket{0}+\beta\ket{1}) = \alpha\ket{0}-\beta\ket{1}
\end{equation}
From \ref{eq:24} we have
\begin{equation}
    \alpha = \alpha, \quad \beta = -\beta \Longrightarrow \beta = 0
\end{equation}
To find $\alpha$ and $\beta$
\begin{equation}
    \|\alpha\|^2 + \|\beta\|^2 = \|\alpha\|^2 + 0 = \|\alpha\|^2 = 1 \Longrightarrow \alpha = 1
\end{equation}
So, $\ket{\gamma_{1}}$ is
\begin{equation}
    \ket{\gamma_{1}} = \ket{0}
\end{equation}
For $\lambda_2 = -1$, from \ref{eq:ref} and \ref{eq:3} we have
\begin{equation}
    \label{eq:28}
    \sigma_{\textnormal{Z}}\ket{\gamma} = \ket{\gamma} \Longrightarrow (\ket{0}\bra{0} - \ket{1}\bra{1})(\alpha\ket{0}+\beta\ket{1}) = \alpha\ket{0}-\beta\ket{1}
\end{equation}
From \ref{eq:28} we have
\begin{equation}
    \alpha = -\alpha \Longrightarrow \alpha = 0, \quad \beta = \beta
\end{equation}
To find $\alpha$ and $\beta$
\begin{equation}
\|\beta\|^2 = \|\beta\|^2 + 0 = \|\beta\|^2 = 1 \Longrightarrow \beta = 1
\end{equation}
So, $\ket{\gamma_{2}}$ is
\begin{equation}
    \ket{\gamma_{2}} = \ket{1}
\end{equation}
\end{document}
