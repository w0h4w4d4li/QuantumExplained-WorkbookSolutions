\documentclass[12pt]{article}
\usepackage{amsmath}  % For advanced math typesetting
\usepackage{amsfonts} % For math fonts
\usepackage{amssymb}  % For additional symbols
\usepackage{graphicx} % For including images
\usepackage{geometry} % For page layout
\usepackage{fancyhdr} % For header and footer
\usepackage{setspace} % For line spacing
\usepackage{hyperref} % For hyperlinks
\usepackage{titlesec} % For customizing section titles
\usepackage{tikz} % For drawing
\usepackage{braket} % For braket notation

\geometry{a4paper, margin=1in}

% Header and Footer
\pagestyle{fancy}
\fancyhf{}
\fancyhead[L]{Chapter 3 - Solutions to Odd-Numbered Exercises}
\fancyhead[R]{\thepage}
\fancyfoot[C]{\thepage}

% Title Formatting
\titleformat{\section}{\Large\bfseries}{\thesection}{1em}{}
\titleformat{\subsection}{\large\bfseries}{\thesubsection}{1em}{}

% Title Page
\title{\textbf{Chapter 3} \\ \small Solutions to Odd-Numbered Exercises}
\author{
    MohamadAli Khajeian\footnote{khajeian@ut.ac.ir} \\ 
    \small \textit{Faculty of Engineering Sciences, University of Tehran, Iran} \\ 
}
\date{\today}

\begin{document}

\maketitle

\begin{abstract}
    This document presents the solution of "Quantum Computing Explained by David McMAHON" exercises.
\end{abstract}

\section*{Exercise 3.1}
\textbf{\( X \) on \( \ket{\psi} \):}
   \[
   X \ket{\psi} = \left( \ket{0} \bra{1} + \ket{1} \bra{0} \right) (\alpha \ket{0} + \beta \ket{1})
   \]
   \[
   X \ket{\psi} = \alpha \ket{1} + \beta \ket{0}
   \]
\textbf{\( Y \) on \( \ket{\psi} \):}
   \[
   Y \ket{\psi} = \left( -i \ket{0} \bra{1} + i \ket{1} \bra{0} \right) (\alpha \ket{0} + \beta \ket{1})
   \]
   \[
   Y \ket{\psi} = -i \beta \ket{0} + i \alpha \ket{1}
   \]
\section*{Exercise 3.3}
   \[
   X \ket{0} = \ket{1}, \quad X \ket{1} = \ket{0}
   \]
\textbf{\( X \) on \( \ket{+} \):}
    \[
    X \ket{+} = X \left( \frac{\ket{0} + \ket{1}}{\sqrt{2}} \right) = \frac{X \ket{0} + X \ket{1}}{\sqrt{2}} = \frac{\ket{1} + \ket{0}}{\sqrt{2}} = \ket{+}
    \]
\textbf{\( X \) on \( \ket{-} \):}
    \[
    X \ket{-} = X \left( \frac{\ket{0} - \ket{1}}{\sqrt{2}} \right) = \frac{X \ket{0} - X \ket{1}}{\sqrt{2}} = \frac{\ket{1} - \ket{0}}{\sqrt{2}} = -\ket{-}
    \]
   Since \( X \ket{+} = \ket{+} \) and \( X \ket{-} = -\ket{-} \),
   \[
   X =  \begin{pmatrix} \braket{+|X|+} & \braket{+|X|-} \\ \braket{-|X|+} & \braket{-|X|-} \end{pmatrix} = \begin{pmatrix} 1 & 0 \\ 0 & -1 \end{pmatrix}
   \]
\section*{Exercise 3.5}
\begin{equation}
    \label{eq:1}
    \sigma_{\textnormal{X}} = \textnormal{X} = 1\ket{0}\bra{1} + 1\ket{1}\bra{0}
\end{equation}
According to
\begin{equation}
    \label{eq:ref}
    \hat{\textnormal{A}}\ket{\gamma} = \lambda\ket{\gamma}
\end{equation}
We assume $\ket{\gamma} = \alpha\ket{0}+\beta\ket{1}$.
Since \ref{eq:1}, we can find eigenvalues through
\begin{equation}
\det{(\sigma_{\textnormal{X}} - \lambda\textnormal{I})} = 0
\end{equation}
\begin{equation*}
\det{\big((-\lambda)\ket{0}\bra{0} + \ket{0}\bra{1} + \ket{1}\bra{0} + (-\lambda)\ket{1}\bra{1}\big)} = 0
\end{equation*}
\begin{equation*}
    \lambda^2 - 1 = 0 \Longrightarrow \lambda = \pm 1
\end{equation*}
For $\lambda_1 = 1$, from \ref{eq:ref} and \ref{eq:1} we have
\begin{equation}
    \label{eq:12}
    \sigma_{\textnormal{X}}\ket{\gamma} = \ket{\gamma} \Longrightarrow (1\ket{0}\bra{1} + 1\ket{1}\bra{0})(\alpha\ket{0}+\beta\ket{1}) = \beta\ket{0}+\alpha\ket{1} 
\end{equation}
From \ref{eq:12} we have
\begin{equation}
    \alpha = \beta
\end{equation}
To find $\alpha$ and $\beta$
\begin{equation}
    \|\alpha\|^2 + \|\beta\|^2 = 1 \Longrightarrow 2\|\alpha\|^2 = 1 \Longrightarrow \alpha = \beta = \frac{1}{\sqrt{2}}
\end{equation}
So, $\ket{\gamma_{1}}$ is
\begin{equation}
    \ket{\gamma_{1}} = \frac{1}{\sqrt{2}}(\ket{0} + \ket{1})
\end{equation}
For $\lambda_2 = -1$, from \ref{eq:ref} and \ref{eq:1} we have
\begin{equation}
    \label{eq:13}
    \sigma_{\textnormal{X}}\ket{\gamma} = -\ket{\gamma} \Longrightarrow (1\ket{0}\bra{1} + 1\ket{1}\bra{0})(\alpha\ket{0}+\beta\ket{1}) = -\beta\ket{0}-\alpha\ket{1} 
\end{equation}
From \ref{eq:13} we have
\begin{equation}
    \alpha = -\beta
\end{equation}
To find $\alpha$ and $\beta$
\begin{equation}
    \|\alpha\|^2 + \|\beta\|^2 = 1 \Longrightarrow 2\|\alpha\|^2 = 1 \Longrightarrow \alpha = - \beta = \frac{1}{\sqrt{2}}
\end{equation}
So, $\ket{\gamma_{2}}$ is
\begin{equation}
    \ket{\gamma_{2}} = \frac{1}{\sqrt{2}}(\ket{0} - \ket{1})
\end{equation}

\section*{Exercise 3.7}
\[
B = \begin{pmatrix} 1 & 0 & 2 \\ 0 & 3 & 4 \\ 1 & 0 & 2 \end{pmatrix}
\]
we need to solve
\[
\det(B - \lambda I) = 0
\]

\[
B - \lambda I = \begin{pmatrix} 1 - \lambda & 0 & 2 \\ 0 & 3 - \lambda & 4 \\ 1 & 0 & 2 - \lambda \end{pmatrix}
\]
by expanding the determinant along the first row, we get
\[
\det(B - \lambda I) = (1 - \lambda) \begin{vmatrix} 3 - \lambda & 4 \\ 0 & 2 - \lambda \end{vmatrix} - 0 + 2 \begin{vmatrix} 0 & 3 - \lambda \\ 1 & 0 \end{vmatrix} = 0
\]
for the first term
\begin{equation*}
(1 - \lambda) \begin{vmatrix} 3 - \lambda & 4 \\ 0 & 2 - \lambda \end{vmatrix} = (1 - \lambda) \cdot (3 - \lambda)(2 - \lambda) = (1 - \lambda)(\lambda^2 - 5\lambda + 6)
\end{equation*}
and for the second term
\begin{equation*}
2 \begin{vmatrix} 0 & 3 - \lambda \\ 1 & 0 \end{vmatrix} = 2 \cdot (\lambda - 3) = 2\lambda - 6    
\end{equation*}
so
\begin{align*}
\det(B - \lambda I) &= (1 - \lambda)(\lambda^2 - 5\lambda + 6) + 2\lambda - 6 \\
&= -\lambda^3 + 6\lambda^2 - 11\lambda + 6 + 2\lambda - 6 \\
&= -\lambda^3 + 6\lambda^2 - 9\lambda \\
&= \lambda(\lambda^2 - 6\lambda + 9) \\
&= \lambda(\lambda - 3)^2 = 0
\end{align*}
The solutions to this equation are
\[
\lambda = 0 \quad \text{and} \quad \lambda = 3 \, (\text{with multiplicity 2})
\]

\section*{Exercise 3.9}
to show that 
\[
X = |0\rangle \langle 1| + |1\rangle \langle 0| = P_+ - P_-
\]
we know
\[
\ket{+} = \frac{1}{\sqrt{2}} (\ket{0} + \ket{1})
\]
\[
\ket{-} = \frac{1}{\sqrt{2}} (\ket{0} - \ket{1})
\]
we need to compute \( P_+ = |+\rangle \langle +| \) and \( P_- = |-\rangle \langle -| \)
\[
P_+ = \left( \frac{1}{\sqrt{2}} (\ket{0} + \ket{1}) \right) \left( \frac{1}{\sqrt{2}} (\bra{0} + \bra{1}) \right) = \frac{1}{2} (|0\rangle \langle 0| + |0\rangle \langle 1| + |1\rangle \langle 0| + |1\rangle \langle 1|).
\]
\[
P_- = \left( \frac{1}{\sqrt{2}} (\ket{0} - \ket{1}) \right) \left( \frac{1}{\sqrt{2}} (\bra{0} - \bra{1}) \right) = \frac{1}{2} (|0\rangle \langle 0| - |0\rangle \langle 1| - |1\rangle \langle 0| + |1\rangle \langle 1|).
\]
now, let's calculate \( P_+ - P_- \)
\[
P_+ - P_- =  \frac{1}{2} (|0\rangle \langle 1| + |0\rangle \langle 1| + |1\rangle \langle 0| + |1\rangle \langle 0|) = |0\rangle \langle 1| + |1\rangle \langle 0|.
\]
thus, we find that
\[
P_+ - P_- = |0\rangle \langle 1| + |1\rangle \langle 0| = X.
\]

\section*{Exercise 3.11}
The Pauli matrices are given as
\begin{align}
\sigma_1 &= \ket{0}\bra{1} + \ket{1}\bra{0}, \label{s1} \\
\sigma_2 &= -i \ket{0}\bra{1} + i \ket{1}\bra{0}, \label{s2} \\
\sigma_3 &= \ket{0}\bra{0} - \ket{1}\bra{1}. \label{s3}
\end{align}

\subsection*{Part 1: Show that \( [ \sigma_2, \sigma_3 ] = 2i \sigma_1 \)}

The commutator \( [ \sigma_2, \sigma_3 ] \) is defined as
\begin{equation}
[ \sigma_2, \sigma_3 ] = \sigma_2 \sigma_3 - \sigma_3 \sigma_2.
\end{equation}
\[
\sigma_2 \sigma_3 = \left( -i \ket{0}\bra{1} + i \ket{1}\bra{0} \right) \left( \ket{0}\bra{0} - \ket{1}\bra{1} \right).
\]
\[
\sigma_2 \sigma_3 = i \ket{0}\bra{1} + i \ket{1}\bra{0}.
\]
similarly, we have
\[
\sigma_3 \sigma_2 = \left( \ket{0}\bra{0} - \ket{1}\bra{1} \right) \left( -i \ket{0}\bra{1} + i \ket{1}\bra{0} \right).
\]
\[
\sigma_3 \sigma_2 = -i \ket{0}\bra{1} - i \ket{1}\bra{0}.
\]
now we subtract \(\sigma_3 \sigma_2\) from \(\sigma_2 \sigma_3\)
\[
[ \sigma_2, \sigma_3 ] = \sigma_2 \sigma_3 - \sigma_3 \sigma_2 = \left( i \ket{0}\bra{1} + i \ket{1}\bra{0} \right) - \left( -i \ket{0}\bra{1} - i \ket{1}\bra{0} \right).
\]
simplifying, we get
\[
[ \sigma_2, \sigma_3 ] = 2i \left( \ket{0}\bra{1} + \ket{1}\bra{0} \right).
\]
since \ref{s1}, we conclude
\begin{equation}
[ \sigma_2, \sigma_3 ] = 2i \sigma_1.
\end{equation}

\subsection*{Part 2: Show that \( [ \sigma_3, \sigma_1 ] = 2i \sigma_2 \)}
The commutator \( [ \sigma_3, \sigma_1 ] \) is defined as
\begin{equation}
    [ \sigma_3, \sigma_1 ] = \sigma_3 \sigma_1 - \sigma_1 \sigma_3.    
\end{equation}
\[
\sigma_3 \sigma_1 = \left( \ket{0}\bra{0} - \ket{1}\bra{1} \right) \left( \ket{0}\bra{1} + \ket{1}\bra{0} \right).
\]
\[
\sigma_3 \sigma_1 = \ket{0}\bra{1} - \ket{1}\bra{0}.
\]
similarly, we have
\[
\sigma_1 \sigma_3 = \left( \ket{0}\bra{1} + \ket{1}\bra{0} \right) \left( \ket{0}\bra{0} - \ket{1}\bra{1} \right).
\]
\[
\sigma_1 \sigma_3 = - \ket{0}\bra{1} + \ket{1}\bra{0}.
\]
now we subtract \(\sigma_1 \sigma_3\) from \(\sigma_3 \sigma_1\)
\[
[ \sigma_3, \sigma_1 ] = \left( \ket{0}\bra{1} - \ket{1}\bra{0} \right) - \left( - \ket{0}\bra{1} + \ket{1}\bra{0} \right).
\]
simplifying, we get
\[
[ \sigma_3, \sigma_1 ] = 2 \left( \ket{0}\bra{1} - \ket{1}\bra{0} \right).
\]
since \ref{s2}, we conclude
\begin{equation}
[ \sigma_3, \sigma_1 ] = 2i \sigma_2.
\end{equation}

\end{document}
