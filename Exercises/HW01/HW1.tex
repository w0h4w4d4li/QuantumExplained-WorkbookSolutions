\documentclass[12pt]{article}
\usepackage{amsmath}  % For advanced math typesetting
\usepackage{amsfonts} % For math fonts
\usepackage{amssymb}  % For additional symbols
\usepackage{graphicx} % For including images
\usepackage{geometry} % For page layout
\usepackage{fancyhdr} % For header and footer
\usepackage{setspace} % For line spacing
\usepackage{hyperref} % For hyperlinks
\usepackage{titlesec} % For customizing section titles
\usepackage{tikz} % For drawing
\usepackage{braket} % For braket notation

\geometry{a4paper, margin=1in}

% Header and Footer
\pagestyle{fancy}
\fancyhf{}
\fancyhead[L]{Chapter 2 - Solutions to Even-Numbered Exercises}
\fancyhead[R]{\thepage}
\fancyfoot[C]{\thepage}

% Title Formatting
\titleformat{\section}{\Large\bfseries}{\thesection}{1em}{}
\titleformat{\subsection}{\large\bfseries}{\thesubsection}{1em}{}

% Title Page
\title{\textbf{Chapter 2} \\ \small Solutions to Even-Numbered Exercises}
\author{
    MohamadAli Khajeian\footnote{khajeian@ut.ac.ir} \\ 
    \small \textit{Faculty of Engineering Sciences, University of Tehran, Iran} \\ 
}
\date{\today}

\begin{document}

\maketitle

\begin{abstract}
    This document presents the solution of "Quantum Computing Explained by David McMAHON" exercises.
\end{abstract}

\section*{Exercise 2.2}
Two quantum states are given by

\begin{center}
    $\left| a \right\rangle = \begin{pmatrix} -4i \\ 2 \end{pmatrix}, \quad$
    $\left| b \right\rangle = \begin{pmatrix} 1 \\ -1 + i \end{pmatrix}$
\end{center}

\subsubsection*{(A)}

\[
|a + b\rangle = -4i|0\rangle + 2|1\rangle + 1|0\rangle + (i - 1)|1\rangle = (-4i+1)|0\rangle + (i+1)|1\rangle
\]

\subsubsection*{(B)}

\[
|a\rangle = \begin{pmatrix} -4i \\ 2 \end{pmatrix}, \quad |b\rangle = \begin{pmatrix} 1 \\ -1 + i \end{pmatrix}
\]

First, we calculate:

\[
3|a\rangle = -12i|0\rangle + 6|1\rangle , \quad 2|b\rangle = 2|0\rangle + (2i - 2)|1\rangle
\]

Now, subtract the two vectors:
\[
3|a\rangle - 2|b\rangle = (-12i - 2)|0\rangle + (8 - 2i)|1\rangle
\]

\subsubsection*{(C)}
\[
\|a\| = \sqrt{(4i\bra{0} + 2\bra{1})(-4i\ket{0} + 2\ket{1})}
\]
\[
= \sqrt{16\braket{0|0}+8i\braket{0|1}-8i\braket{1|0}+4\braket{1|1}} =  \sqrt{16 + 4} = \sqrt{20} \Longrightarrow \ket{a} = \frac{1}{\sqrt{20}}(-4i\ket{0} + 2\ket{1})
\]
\[
\|b\| = \sqrt{(1\bra{0} + (-1-i)\bra{1})(1\ket{0} + (-1+i)\ket{1})}
\]
\[
= \sqrt{1\braket{0|0}+(-1+i)\braket{0|1}+(-1-i)\braket{1|0}+2\braket{1|1}} =  \sqrt{2 + 1} = \sqrt{3} \Longrightarrow \ket{b} = \frac{1}{\sqrt{3}}(1\ket{0} + (-1+i)\ket{1})
\]

\section*{Exercise 2.4}
A quantum system is in the state

\[
\ket{\psi} = \frac{3i\ket{0} + 4\ket{1}}{5}
\]

\subsubsection*{(A)}

Yes, because we have

\[
\ket{\psi} = \frac{1}{5}(3i\ket{0} + 4\ket{1})
\]

\[
\|\psi\| = \sqrt{\frac{1}{25}(-3i\bra{0} + 4\bra{1})(3i\ket{0} + 4\ket{1})}
= \frac{1}{5}\sqrt{(-3i\bra{0} + 4\bra{1})(3i\ket{0} + 4\ket{1})}
\]

\[
= \frac{1}{5}\sqrt{9\braket{0|0} - 12i\braket{0|1} + 12i\braket{1|0} + 16\braket{1|1}}
= \frac{1}{5}\sqrt{9+16} = \frac{1}{5}\sqrt{25} = 1
\]

\subsubsection*{(B)}
Now, we know
\begin{equation}
\label{eq:1}
\ket{\psi} = (\frac{3i}{5}\ket{0} + \frac{4}{5}\ket{1})
\end{equation}
We need to find $\alpha$ and $\beta$ such that satisfy
\[
\ket{\phi} = \alpha \ket{+} + \beta \ket{-}
\]
Where
\[
\ket{+} = \frac{\ket{0} + \ket{1}}{\sqrt{2}}, \quad \ket{-} = \frac{\ket{0} - \ket{1}}{\sqrt{2}}
\]
So
\begin{equation}
\label{eq:2}
\ket{\phi} = \alpha (\frac{\ket{0} + \ket{1}}{\sqrt{2}}) + \beta (\frac{\ket{0} - \ket{1}}{\sqrt{2}})
= (\frac{\alpha + \beta}{\sqrt{2}})\ket{0} + (\frac{\alpha - \beta}{\sqrt{2}})\ket{1} 
\end{equation}
From (\ref{eq:1}) and (\ref{eq:2})
\begin{equation}
\label{eq:3}
\frac{\alpha + \beta}{\sqrt{2}} = \frac{3i}{5} \Longrightarrow \alpha + \beta = \frac{3\sqrt{2}i}{5}
\end{equation}
\begin{equation}
\label{eq:4}
\frac{\alpha - \beta}{\sqrt{2}} = \frac{4}{5} \Longrightarrow \alpha - \beta = \frac{4\sqrt{2}}{5} 
\end{equation}
If we calculate (\ref{eq:3}) + (\ref{eq:4})
\begin{equation*}
2\alpha = \frac{3\sqrt{2}i}{5} + \frac{4\sqrt{2}}{5} = \frac{\sqrt{2}}{5}(3i + 4) \Longrightarrow \alpha = \frac{\sqrt{2}}{10}(3i + 4)
\end{equation*}
Also we can get $\beta$ using \ref{eq:3},
\begin{equation*}
\beta = \frac{\sqrt{2}}{5}(3i) - \frac{\sqrt{2}}{10}(3i + 4) = \frac{\sqrt{2}}{10}(6i - 3i - 4) \Longrightarrow \beta = \frac{\sqrt{2}}{10}(3i - 4)
\end{equation*}
Finally we have 
\begin{equation*}
\ket{\phi} = (\frac{\sqrt{2}}{10}(3i + 4)) \ket{+} + (\frac{\sqrt{2}}{10}(3i - 4)) \ket{-}
\end{equation*}

\section*{Exercise 2.6}
Photon horizontal and vertical polarization states are written as $\ket{h}$ and $\ket{v}$,
respectively. Suppose

\begin{equation*}
\ket{\psi_1}  = \frac{1}{2}\ket{h} + \frac{\sqrt{3}}{2}\ket{v}
\end{equation*}
\begin{equation*}
\ket{\psi_2}  = \frac{1}{2}\ket{h} - \frac{\sqrt{3}}{2}\ket{v}
\end{equation*}
\begin{equation*}
\ket{\psi_3}  = \ket{h}
\end{equation*}
\\
We can get $\bra{\psi_1}$ and $\bra{\psi_3}$
\begin{equation*}
\bra{\psi_1}  = \frac{1}{2}\bra{h} + \frac{\sqrt{3}}{2}\bra{v}
\end{equation*}
\begin{equation*}
\bra{\psi_3}  = \bra{h}
\end{equation*}
\\
Now, since $\braket{v|h}=\braket{h|v}=0$ and $\braket{v|v}=\braket{h|h}=1$, We can calculate $|\braket{\psi_1|\psi_2}|^{2}$, $|\braket{\psi_1|\psi_3}|^2$ and $|\braket{\psi_3|\psi_2}|^2$,
\begin{equation*}
\braket{\psi_1|\psi_2} = (\frac{1}{2}\bra{h} + \frac{\sqrt{3}}{2}\bra{v})(\frac{1}{2}\ket{h} - \frac{\sqrt{3}}{2}\ket{v})
= \frac{1}{4}\braket{h|h} - \frac{\sqrt{3}}{4}\braket{h|v} + \frac{\sqrt{3}}{4}\braket{v|h} - \frac{3}{4}\braket{v|v}
= -\frac{1}{2}
\end{equation*}
\begin{equation*}
|\braket{\psi_1|\psi_2}|^2 =  \frac{1}{4},
\end{equation*}
\begin{equation*}
\braket{\psi_1|\psi_3} = (\frac{1}{2}\bra{h} + \frac{\sqrt{3}}{2}\bra{v})(\ket{h}) = \frac{1}{2}\braket{h|h} + \frac{\sqrt{3}}{2}\braket{v|h}
= \frac{1}{2}
\end{equation*}
\begin{equation*}
|\braket{\psi_1|\psi_3}|^2 =  \frac{1}{4},
\end{equation*}
\begin{equation*}
\braket{\psi_3|\psi_2} = (\bra{h})(\frac{1}{2}\ket{h} - \frac{\sqrt{3}}{2}\ket{v}) = \frac{1}{2}\braket{h|h} - \frac{\sqrt{3}}{2}\braket{h|v}
= \frac{1}{2}
\end{equation*}
\begin{equation*}
|\braket{\psi_3|\psi_2}|^2 =  \frac{1}{4}.
\end{equation*}
\end{document}
