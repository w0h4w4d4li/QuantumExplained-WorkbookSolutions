\documentclass[12pt]{article}
\usepackage{amsmath}  % For advanced math typesetting
\usepackage{amsfonts} % For math fonts
\usepackage{amssymb}  % For additional symbols
\usepackage{graphicx} % For including images
\usepackage{geometry} % For page layout
\usepackage{fancyhdr} % For header and footer
\usepackage{setspace} % For line spacing
\usepackage{hyperref} % For hyperlinks
\usepackage{titlesec} % For customizing section titles
\usepackage{tikz} % For drawing
\usepackage{braket} % For braket notation

\geometry{a4paper, margin=1in}

% Header and Footer
\pagestyle{fancy}
\fancyhf{}
\fancyhead[L]{Chapter 3 - Solutions to Try it}
\fancyhead[R]{\thepage}
\fancyfoot[C]{\thepage}

% Title Formatting
\titleformat{\section}{\Large\bfseries}{\thesection}{1em}{}
\titleformat{\subsection}{\large\bfseries}{\thesubsection}{1em}{}

% Title Page
\title{\textbf{Chapter 3} \\ \small Solutions to Try it}
\author{
    MohamadAli Khajeian\footnote{khajeian@ut.ac.ir} \\ 
    \small \textit{Faculty of Engineering Sciences, University of Tehran, Iran} \\ 
}
\date{\today}

\begin{document}

\maketitle

\begin{abstract}
    This document presents the solution of "Quantum Computing Explained by David McMAHON" exercises.
\end{abstract}

\section*{Try it - (page 42)}
Let an arbitrary qubit state be given by
\[
|\psi\rangle = \alpha |0\rangle + \beta |1\rangle,
\]
now, apply the operator \(\hat{A} = |0\rangle\langle 0| + |1\rangle\langle 1|\) to this state
\[
\hat{A} |\psi\rangle = (|0\rangle\langle 0| + |1\rangle\langle 1|)(\alpha |0\rangle + \beta |1\rangle).
\]
\[
\hat{A} |\psi\rangle = |0\rangle\langle 0|(\alpha |0\rangle + \beta |1\rangle) + |1\rangle\langle 1|(\alpha |0\rangle + \beta |1\rangle).
\]
we get
\[
\hat{A} |\psi\rangle = \alpha |0\rangle + \beta |1\rangle = |\psi\rangle.
\]

\section*{Try it - (page 44)}
Let
\begin{equation*}
    \ket{\psi} = a\ket{1} + b\ket{2} + c\ket{3}, \quad \ket{\phi} = e\ket{1} + f\ket{2} + g\ket{3}
\end{equation*}
we can calculate outer product
\begin{align*}
    |\psi\rangle \langle\phi| &= (a\ket{1} + b\ket{2} + c\ket{3})(e^*\bra{1} + f^*\bra{2} + g^*\bra{3}) \\
    &= ae^{*}|1\rangle \langle1| + af^{*}|1\rangle \langle2| + ag^{*}|1\rangle \langle3| + be^{*}|2\rangle \langle1| + bf^{*}|2\rangle \langle2| \\
    &+ bg^{*}|2\rangle \langle3| + ce^{*}|3\rangle \langle1| + cf^{*}|3\rangle \langle2| + cg^{*}|3\rangle \langle3|
\end{align*}

\section*{Try it - (page 45)}
since 
\begin{equation*}
    \sigma_{0}\ket{0} = \ket{0}, \quad \sigma_{0}\ket{1} = \ket{1}  
\end{equation*}
we have
\begin{equation*}
    \sigma_{0} = \begin{pmatrix} \braket{0|\sigma_{0}|0} & \braket{0|\sigma_{0}|1} \\ \braket{1|\sigma_{0}|0} & \braket{1|\sigma_{0}|1} \end{pmatrix}
    = \begin{pmatrix} 1 & 0 \\ 0 & 1 \end{pmatrix}
\end{equation*}

\section*{Try it - (page 47)}
Let
\begin{equation*}
    \hat{B} = 3i|0\rangle \langle0| + 2i|0\rangle \langle1|
\end{equation*}
we can compute the adjoint of each term 
\begin{align*}
    \hat{B}^{\dagger} &= (3i|0\rangle \langle0|)^{\dagger} + (2i|0\rangle \langle1|)^{\dagger} \\
    &= -3i|0\rangle \langle0| - 2i|1\rangle \langle0|
\end{align*}

\section*{Try it - (page 50)}
According to
\begin{equation}
    \label{eq:ref}
    \hat{\textnormal{A}}\ket{\gamma} = \lambda\ket{\gamma}
\end{equation}
We assume $\ket{\gamma} = \alpha\ket{0}+\beta\ket{1}$.
\begin{equation}
    \label{eq:3}
    \sigma_{\textnormal{Z}} = \textnormal{Z} = 1\ket{0}\bra{0} -1\ket{1}\bra{1}
\end{equation}
Since \ref{eq:3}, we can find eigenvalues through
\begin{equation}
\det{(\sigma_{\textnormal{Z}} - \lambda\textnormal{I})} = 0
\end{equation}
\begin{equation*}
\det{\big((1-\lambda)\ket{0}\bra{0} + (-1-\lambda)\ket{1}\bra{1}\big)} = 0
\end{equation*}
\begin{equation*}
    (1-\lambda)(-1-\lambda) = -1 - \lambda + \lambda + \lambda^2 = \lambda^2 -1 = 0 \Longrightarrow \lambda = \pm 1
\end{equation*}
For $\lambda_1 = 1$, from \ref{eq:ref} and \ref{eq:3} we have
\begin{equation}
    \label{eq:24}
    \sigma_{\textnormal{Z}}\ket{\gamma} = \ket{\gamma} \Longrightarrow (\ket{0}\bra{0} - \ket{1}\bra{1})(\alpha\ket{0}+\beta\ket{1}) = \alpha\ket{0}-\beta\ket{1}
\end{equation}
From \ref{eq:24} we have
\begin{equation}
    \alpha = \alpha, \quad \beta = -\beta \Longrightarrow \beta = 0
\end{equation}
To find $\alpha$ and $\beta$
\begin{equation}
    \|\alpha\|^2 + \|\beta\|^2 = \|\alpha\|^2 + 0 = \|\alpha\|^2 = 1 \Longrightarrow \alpha = 1
\end{equation}
So, $\ket{\gamma_{1}}$ is
\begin{equation}
    \ket{\gamma_{1}} = \ket{0}
\end{equation}
For $\lambda_2 = -1$, from \ref{eq:ref} and \ref{eq:3} we have
\begin{equation}
    \label{eq:28}
    \sigma_{\textnormal{Z}}\ket{\gamma} = \ket{\gamma} \Longrightarrow (\ket{0}\bra{0} - \ket{1}\bra{1})(\alpha\ket{0}+\beta\ket{1}) = \alpha\ket{0}-\beta\ket{1}
\end{equation}
From \ref{eq:28} we have
\begin{equation}
    \alpha = -\alpha \Longrightarrow \alpha = 0, \quad \beta = \beta
\end{equation}
To find $\alpha$ and $\beta$
\begin{equation}
\|\beta\|^2 = \|\beta\|^2 + 0 = \|\beta\|^2 = 1 \Longrightarrow \beta = 1
\end{equation}
So, $\ket{\gamma_{2}}$ is
\begin{equation}
    \ket{\gamma_{2}} = \ket{1}
\end{equation}

\section*{Try it - (page 63)}
to show that 
\[
P_+ + P_- = |0\rangle \langle 0| + |1\rangle \langle 1| = I
\]
we know
\[
\ket{+} = \frac{1}{\sqrt{2}} (\ket{0} + \ket{1})
\]
\[
\ket{-} = \frac{1}{\sqrt{2}} (\ket{0} - \ket{1})
\]
we need to compute \( P_+ = |+\rangle \langle +| \) and \( P_- = |-\rangle \langle -| \)
\[
P_+ = \left( \frac{1}{\sqrt{2}} (\ket{0} + \ket{1}) \right) \left( \frac{1}{\sqrt{2}} (\bra{0} + \bra{1}) \right) = \frac{1}{2} (|0\rangle \langle 0| + |0\rangle \langle 1| + |1\rangle \langle 0| + |1\rangle \langle 1|).
\]
\[
P_- = \left( \frac{1}{\sqrt{2}} (\ket{0} - \ket{1}) \right) \left( \frac{1}{\sqrt{2}} (\bra{0} - \bra{1}) \right) = \frac{1}{2} (|0\rangle \langle 0| - |0\rangle \langle 1| - |1\rangle \langle 0| + |1\rangle \langle 1|).
\]
now, let's calculate \( P_+ - P_- \)
\[
P_+ + P_- =  \frac{1}{2} (|0\rangle \langle 0| + |0\rangle \langle 0| + |1\rangle \langle 1| + |1\rangle \langle 1|) = |0\rangle \langle 0| + |1\rangle \langle 1|.
\]
thus, we find that
\[
P_+ = P_- = |0\rangle \langle 0| + |1\rangle \langle 1| = I.
\]

\end{document}
