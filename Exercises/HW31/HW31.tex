\documentclass{article}
\usepackage{amsmath}  % For advanced math typesetting
\usepackage{amsfonts} % For math fonts
\usepackage{amssymb}  % For additional symbols
\usepackage{graphicx} % For including images
\usepackage{geometry} % For page layout
\usepackage{fancyhdr} % For header and footer
\usepackage{setspace} % For line spacing
\usepackage{hyperref} % For hyperlinks
\usepackage{titlesec} % For customizing section titles
\usepackage{tikz} % For drawing
\usepackage{braket} % For braket notation

\geometry{a4paper, margin=1in}

% Header and Footer
\pagestyle{fancy}
\fancyhf{}
\fancyhead[L]{Chapter 8 - Solutions to Even-Numbered Exercises}
\fancyhead[R]{\thepage}
\fancyfoot[C]{\thepage}

% Title Formatting
\titleformat{\section}{\Large\bfseries}{\thesection}{1em}{}
\titleformat{\subsection}{\large\bfseries}{\thesubsection}{1em}{}

% Title Page
\title{\textbf{Chapter 8} \\ \small Solutions to Even-Numbered Exercises}
\author{
    MohamadAli Khajeian\footnote{khajeian@ut.ac.ir} \\ 
    \small \textit{Faculty of Engineering Sciences, University of Tehran, Iran} \\ 
}
\date{\today}

% Commands
\newcommand{\op}[2]{|#1\rangle \langle#2|}
\newcommand{\sand}[3]{\braket{#1 | #2 | #3}}
\newcommand{\sandop}[3]{\braket{#1 #2 #3}}
\newcommand{\tensor}[2]{#1 \otimes #2}

\begin{document}

\maketitle

\begin{abstract}
    This document presents the solution of "Quantum Computing Explained by David McMAHON" exercises.
\end{abstract}

\section*{Exercise 8.2}

\textbf{(A) Matrix representations of the Hubbard operators in the computational basis:}

\[
X^{11} = |0\rangle\langle 0|, \quad
X^{12} = |0\rangle\langle 1|, \quad
X^{21} = |1\rangle\langle 0|, \quad
X^{22} = |1\rangle\langle 1|.
\]
\[
X^{11} = \begin{pmatrix} 1 & 0 \\ 0 & 0 \end{pmatrix}, \quad
X^{12} = \begin{pmatrix} 0 & 1 \\ 0 & 0 \end{pmatrix}, \quad
X^{21} = \begin{pmatrix} 0 & 0 \\ 1 & 0 \end{pmatrix}, \quad
X^{22} = \begin{pmatrix} 0 & 0 \\ 0 & 1 \end{pmatrix}.
\]
\textbf{(B) Action of the Hubbard operators on the Hadamard basis states:}
\[
|+\rangle = \frac{1}{\sqrt{2}} \big(|0\rangle + |1\rangle \big), \quad
|-\rangle = \frac{1}{\sqrt{2}} \big(|0\rangle - |1\rangle \big).
\]

\begin{itemize}
    \item \textbf{Action of \(X^{11}\):}
    \[
    X^{11}|+\rangle = |0\rangle\langle 0| \cdot \frac{1}{\sqrt{2}} \big(|0\rangle + |1\rangle \big) = \frac{1}{\sqrt{2}}|0\rangle,
    \]
    \[
    X^{11}|-\rangle = |0\rangle\langle 0| \cdot \frac{1}{\sqrt{2}} \big(|0\rangle - |1\rangle \big) = \frac{1}{\sqrt{2}}|0\rangle.
    \]

    \item \textbf{Action of \(X^{12}\):}
    \[
    X^{12}|+\rangle = |0\rangle\langle 1| \cdot \frac{1}{\sqrt{2}} \big(|0\rangle + |1\rangle \big) = \frac{1}{\sqrt{2}}|0\rangle,
    \]
    \[
    X^{12}|-\rangle = |0\rangle\langle 1| \cdot \frac{1}{\sqrt{2}} \big(|0\rangle - |1\rangle \big) = -\frac{1}{\sqrt{2}}|0\rangle.
    \]

    \item \textbf{Action of \(X^{21}\):}
    \[
    X^{21}|+\rangle = |1\rangle\langle 0| \cdot \frac{1}{\sqrt{2}} \big(|0\rangle + |1\rangle \big) = \frac{1}{\sqrt{2}}|1\rangle,
    \]
    \[
    X^{21}|-\rangle = |1\rangle\langle 0| \cdot \frac{1}{\sqrt{2}} \big(|0\rangle - |1\rangle \big) = \frac{1}{\sqrt{2}}|1\rangle.
    \]

    \item \textbf{Action of \(X^{22}\):}
    \[
    X^{22}|+\rangle = |1\rangle\langle 1| \cdot \frac{1}{\sqrt{2}} \big(|0\rangle + |1\rangle \big) = \frac{1}{\sqrt{2}}|1\rangle,
    \]
    \[
    X^{22}|-\rangle = |1\rangle\langle 1| \cdot \frac{1}{\sqrt{2}} \big(|0\rangle - |1\rangle \big) = -\frac{1}{\sqrt{2}}|1\rangle.
    \]
\end{itemize}
\section*{Exercise 8.4}

\[
\text{CNOT} = \begin{pmatrix}
1 & 0 & 0 & 0 \\
0 & 1 & 0 & 0 \\
0 & 0 & 0 & 1 \\
0 & 0 & 1 & 0
\end{pmatrix}.
\]

\subsection*{1. Hermitian Property}

A matrix \(A\) is Hermitian if \(A^\dagger = A\), where \(A^\dagger\) is the conjugate transpose of \(A\). For the CNOT gate:
\[
\text{CNOT}^\dagger = \begin{pmatrix}
1 & 0 & 0 & 0 \\
0 & 1 & 0 & 0 \\
0 & 0 & 0 & 1 \\
0 & 0 & 1 & 0
\end{pmatrix}.
\]
Since \(\text{CNOT}^\dagger = \text{CNOT}\), the CNOT gate is Hermitian.

\subsection*{2. Unitary Property}

A matrix \(A\) is unitary if \(A^\dagger A = I\), where \(I\) is the identity matrix. For the CNOT gate:
\[
\text{CNOT} \cdot \text{CNOT} = \begin{pmatrix}
1 & 0 & 0 & 0 \\
0 & 1 & 0 & 0 \\
0 & 0 & 0 & 1 \\
0 & 0 & 1 & 0
\end{pmatrix}
\cdot
\begin{pmatrix}
1 & 0 & 0 & 0 \\
0 & 1 & 0 & 0 \\
0 & 0 & 0 & 1 \\
0 & 0 & 1 & 0
\end{pmatrix}.
\]
Performing the matrix multiplication:
\[
\text{CNOT} \cdot \text{CNOT} = \begin{pmatrix}
1 & 0 & 0 & 0 \\
0 & 1 & 0 & 0 \\
0 & 0 & 1 & 0 \\
0 & 0 & 0 & 1
\end{pmatrix} = I.
\]
Since \(\text{CNOT}^\dagger \cdot \text{CNOT} = I\), the CNOT gate is unitary.

\section*{Exercise 8.6}
\begin{align*}
    \ket{00} &\rightarrow \ket{00} \\
    \ket{01} &\rightarrow \ket{01} \\
    \ket{10} &\rightarrow \ket{10} \\
    \ket{11} &\rightarrow -\ket{11} 
\end{align*}

\subsection*{Matrix Representation}
\[
CZ =
\begin{bmatrix}
1 & 0 & 0 & 0 \\
0 & 1 & 0 & 0 \\
0 & 0 & 1 & 0 \\
0 & 0 & 0 & -1
\end{bmatrix}.
\]

\subsection*{Dirac Notation}
\[
CZ = |00\rangle \langle 00| + |01\rangle \langle 01| + |10\rangle \langle 10| - |11\rangle \langle 11|.
\]

\section*{Exercise 8.8}

A rotation by an angle \(\theta\) around an axis defined by the unit vector \(\mathbf{n} = (n_x, n_y, n_z)\) is given by the rotation operator:
\[
R_{\mathbf{n}}(\theta) = \exp\left(-i \frac{\theta}{2} \mathbf{n} \cdot \mathbf{\sigma} \right),
\]
where \(\mathbf{\sigma} = (\sigma_x, \sigma_y, \sigma_z)\) are the Pauli matrices.
For this problem:
\[
\mathbf{n} = \frac{\mathbf{e}_x + \mathbf{e}_z}{\sqrt{2}} = \frac{1}{\sqrt{2}} (1, 0, 1),
\]
and the rotation angle is \(\theta = \pi\) (180 degrees).
Substituting \(\mathbf{n}\) and \(\theta\) into the rotation operator:
\[
R_{\mathbf{n}}(\pi) = \exp\left(-i \frac{\pi}{2} \frac{\sigma_x + \sigma_z}{\sqrt{2}} \right).
\]
The Pauli matrices \(\sigma_x\) and \(\sigma_z\) are:
\[
\sigma_x = \begin{bmatrix} 0 & 1 \\ 1 & 0 \end{bmatrix}, \quad \sigma_z = \begin{bmatrix} 1 & 0 \\ 0 & -1 \end{bmatrix}.
\]
Thus:
\[
\frac{\sigma_x + \sigma_z}{\sqrt{2}} = \frac{1}{\sqrt{2}}\begin{bmatrix} 1 & 1 \\ 1 & -1 \end{bmatrix}.
\]
The exponential operator \(\exp\left(-i \frac{\pi}{2} \cdot \frac{\sigma_x + \sigma_z}{\sqrt{2}} \right)\) simplifies to the Hadamard gate:
\begin{align*}
    \exp\big(-i \frac{\pi}{2} \cdot \frac{\sigma_x + \sigma_z}{\sqrt{2}}\big) &= \cos\frac{\pi}{2}I - i \sin\frac{\pi}{2}\big(\frac{\sigma_x + \sigma_z}{\sqrt{2}}\big) \\
    &= iH
\end{align*}
with factor phase we have
\[
H = \frac{1}{\sqrt{2}} \begin{bmatrix} 1 & 1 \\ 1 & -1 \end{bmatrix}.
\]

\section*{Exercise 8.10}

\[
P_0 \otimes I =
\begin{bmatrix} 1 & 0 \\ 0 & 0 \end{bmatrix} \otimes \begin{bmatrix} 1 & 0 \\ 0 & 1 \end{bmatrix} =
\begin{bmatrix}
1 & 0 & 0 & 0 \\
0 & 1 & 0 & 0 \\
0 & 0 & 0 & 0 \\
0 & 0 & 0 & 0
\end{bmatrix}.
\]

\[
P_1 \otimes X =
\begin{bmatrix} 0 & 0 \\ 0 & 1 \end{bmatrix} \otimes \begin{bmatrix} 0 & 1 \\ 1 & 0 \end{bmatrix} =
\begin{bmatrix}
0 & 0 & 0 & 0 \\
0 & 0 & 0 & 0 \\
0 & 0 & 0 & 1 \\
0 & 0 & 1 & 0
\end{bmatrix}.
\]

\[
CNOT = P_0 \otimes I + P_1 \otimes X =
\begin{bmatrix}
1 & 0 & 0 & 0 \\
0 & 1 & 0 & 0 \\
0 & 0 & 0 & 1 \\
0 & 0 & 1 & 0
\end{bmatrix}.
\]
This is the standard matrix representation of the controlled-NOT gate.


\end{document}
