\documentclass{article}
\usepackage{amsmath}  % For advanced math typesetting
\usepackage{amsfonts} % For math fonts
\usepackage{amssymb}  % For additional symbols
\usepackage{graphicx} % For including images
\usepackage{geometry} % For page layout
\usepackage{fancyhdr} % For header and footer
\usepackage{setspace} % For line spacing
\usepackage{hyperref} % For hyperlinks
\usepackage{titlesec} % For customizing section titles
\usepackage{tikz} % For drawing
\usepackage{braket} % For braket notation
\usepackage{cleveref}

\geometry{a4paper, margin=1in}

% Header and Footer
\pagestyle{fancy}
\fancyhf{}
\fancyhead[L]{Chapter 6 - Solutions to Try it}
\fancyhead[R]{\thepage}
\fancyfoot[C]{\thepage}

% Title Formatting
\titleformat{\section}{\Large\bfseries}{\thesection}{1em}{}
\titleformat{\subsection}{\large\bfseries}{\thesubsection}{1em}{}

% Title Page
\title{\textbf{Chapter 6} \\ \small Solutions to Try it}
\author{
    MohamadAli Khajeian\footnote{khajeian@ut.ac.ir} \\ 
    \small \textit{Faculty of Engineering Sciences, University of Tehran, Iran} \\ 
}
\date{\today}

% Commands
\newcommand{\op}[2]{|#1\rangle \langle#2|}
\newcommand{\sand}[3]{\braket{#1 | #2 | #3}}
\newcommand{\sandop}[3]{\braket{#1 #2 #3}}

\begin{document}

\maketitle

\begin{abstract}
    This document presents the solution of "Quantum Computing Explained by David McMAHON" exercises.
\end{abstract}

\section*{Try it - (page 103)}
we have
\begin{align}
   \text{Y} \ket{u_i} = \lambda_i \ket{u_i}
   \label{1}
\end{align}
where 
\begin{align}
   \ket{u_i} = \alpha \ket{0} + \beta \ket{1}
   \label{2}
\end{align}
let's consider
\begin{align}
   \sigma_y = \text{Y} = 
   \begin{pmatrix}
      0 & -i \\
      i & 0 
   \end{pmatrix}
   = -i\op{0}{1} + i\op{1}{0}
   \label{3}
\end{align}
to get EigenVectors and EigenValues of $\text{Y}$ matrix, we have
\begin{align*}
   \det|\text{Y} - \lambda\text{I}| &= \det\bigg|-\lambda\op{0}{0}-i\op{0}{1} + i\op{1}{0}-\lambda\op{1}{1}\bigg| \\
   &= (-\lambda)(-\lambda) - (-i)(i) \\
   &= \lambda^2 - 1 = 0
\end{align*}
so
\begin{align*}
   \lambda_{1,2} = \pm 1
\end{align*}
now to calculate $\ket{u_1}$, using \cref{1,2,3}
\begin{align*}
   \text{Y} \ket{u_1} &= \lambda_i \ket{u_1} \\
   &=(-i\ket{0}\bra{1} + i\ket{1}\bra{0})(\alpha\ket{0}+\beta\ket{1}) \\
   &= i(-\beta\ket{0}+\alpha\ket{1}) = -\ket{u_1}
\end{align*}
from \ref{2} we have
\begin{align*}
    \alpha = -i\beta, \quad \beta = i\alpha
\end{align*}
to find $\alpha$ and $\beta$
\begin{align*}
    \|\alpha\|^2 + \|\beta\|^2 = \|\alpha\|^2 + (-i\alpha^{*})(i\alpha) = 2\|\alpha\|^2 = 1 \Longrightarrow \alpha = \frac{1}{\sqrt{2}}
\end{align*}
so, $\ket{u_{1}}$ is
\begin{align*}
    \boxed{\ket{u_{1}} = \frac{1}{\sqrt{2}}(\ket{0} + i\ket{1})}
\end{align*}
to calculate $\ket{u_2}$, using \cref{1,2,3}
\begin{align*}
   \text{Y} \ket{u_2} &= \lambda_i \ket{u_2} \\
   &=(-i\ket{0}\bra{1} + i\ket{1}\bra{0})(\alpha\ket{0}+\beta\ket{1}) \\&
   = i(-\beta\ket{0}+\alpha\ket{1}) = -\ket{u_2}
\end{align*}
from \ref{2} we have
\begin{align*}
    \alpha = i\beta, \quad \beta = -i\alpha
\end{align*}
To find $\alpha$ and $\beta$
\begin{align*}
    \|\alpha\|^2 + \|\beta\|^2 = \|\alpha\|^2 + (-i\alpha^{*})(i\alpha) = 2\|\alpha\|^2 = 1 \Longrightarrow \alpha = \frac{1}{\sqrt{2}}
\end{align*}
So, $\ket{u_{2}}$ is
\begin{align*}
    \boxed{\ket{u_{2}} = \frac{1}{\sqrt{2}}(\ket{0} - i\ket{1})}
\end{align*}
\section*{Try it - (page 136)}
if we have
\begin{align*}
   \ket{\psi^{'}} = \bigg(\frac{\sqrt{2}+i}{\sqrt{15}}\bigg)\ket{000} + \sqrt{\frac{2}{3}}\ket{001} + \sqrt{\frac{2}{15}}\ket{011}
\end{align*}
to determine if the state is normalized, we compute the sum of the squares of the coefficients
\begin{align*}
   \sum_i \|c_i\|^2 &= \bigg(\frac{\sqrt{2}-i}{\sqrt{15}}\bigg)\bigg(\frac{\sqrt{2}+i}{\sqrt{15}}\bigg) + \bigg(\sqrt{\frac{2}{3}}\bigg)\bigg(\sqrt{\frac{2}{3}}\bigg) + \bigg(\sqrt{\frac{2}{15}}\bigg)\bigg(\sqrt{\frac{2}{15}}\bigg)\\
   &= \frac{3}{15} + \frac{2}{3} + \frac{2}{15} = \frac{15}{15} = 1
\end{align*}
therefore the state is normalized.
\section*{Try it - (page 138)}
A system is in the GHZ state where
\begin{align}
   \ket{\psi} = \frac{1}{\sqrt{2}}\big(\ket{000}+\ket{111}\big)
   \label{4}
\end{align}
we know
\begin{align}
   \ket{0} &= \frac{1}{\sqrt{2}}\big(\ket{+}+\ket{-}\big) \label{5}\\
   \ket{1} &= \frac{1}{\sqrt{2}}\big(\ket{+}-\ket{-}\big) \label{6}
\end{align}
so using \cref{5,6}
\begin{align*}
   \ket{000} &= \bigg(\frac{1}{\sqrt{2}}\big(\ket{+}+\ket{-}\big)\bigg)\otimes\bigg(\frac{1}{\sqrt{2}}\big(\ket{+}+\ket{-}\big)\bigg)\otimes\bigg(\frac{1}{\sqrt{2}}\big(\ket{+}+\ket{-}\big)\bigg)\\
   &= \bigg(\frac{1}{\sqrt{2}}\big(\ket{+}+\ket{-}\big)\bigg)\otimes
   \frac{1}{2}\bigg(\ket{++}+\ket{+-}+\ket{-+}+\ket{--}\bigg) \\
   &= \frac{1}{2\sqrt{2}}\bigg(\ket{+++}+\ket{++-}+\ket{+-+}+\ket{+--}+\ket{-++}+\ket{-+-}+\ket{--+}+\ket{---}\bigg)
\end{align*}
\begin{align*}
   \ket{111} &= \bigg(\frac{1}{\sqrt{2}}\big(\ket{+}-\ket{-}\big)\bigg)\otimes\bigg(\frac{1}{\sqrt{2}}\big(\ket{+}-\ket{-}\big)\bigg)\otimes\bigg(\frac{1}{\sqrt{2}}\big(\ket{+}-\ket{-}\big)\bigg)\\
   &= \bigg(\frac{1}{\sqrt{2}}\big(\ket{+}-\ket{-}\big)\bigg)\otimes
   \frac{1}{2}\bigg(\ket{++}-\ket{+-}-\ket{-+}+\ket{--}\bigg) \\
   &= \frac{1}{2\sqrt{2}}\bigg(\ket{+++}-\ket{++-}-\ket{+-+}+\ket{+--}-\ket{-++}+\ket{-+-}+\ket{--+}-\ket{---}\bigg)
\end{align*}
now using \cref{4} we have
\begin{align*}
   \ket{\psi} &= \frac{1}{\sqrt{2}}\big(\ket{000}+\ket{111}\big)\\
   &= \frac{1}{\sqrt{2}}\bigg(\frac{1}{2\sqrt{2}}\bigg(\ket{+++}+\ket{++-}+\ket{+-+}+\ket{+--}+\ket{-++}+\ket{-+-}+\ket{--+}+\ket{---}\bigg) \\
   &+ \frac{1}{2\sqrt{2}}\bigg(\ket{+++}-\ket{++-}-\ket{+-+}+\ket{+--}-\ket{-++}+\ket{-+-}+\ket{--+}-\ket{---}\bigg)\bigg) \\
   &= \frac{1}{4}\bigg(2\ket{+++}+2\ket{+--}+2\ket{-+-}+2\ket{--+}\bigg) \\
   &= \frac{1}{2}\bigg(\ket{+++}+\ket{+--}+\ket{-+-}+\ket{--+}\bigg).
\end{align*}
\end{document}
