\documentclass[12pt]{article}
\usepackage{amsmath}  % For advanced math typesetting
\usepackage{amsfonts} % For math fonts
\usepackage{amssymb}  % For additional symbols
\usepackage{graphicx} % For including images
\usepackage{geometry} % For page layout
\usepackage{fancyhdr} % For header and footer
\usepackage{setspace} % For line spacing
\usepackage{hyperref} % For hyperlinks
\usepackage{titlesec} % For customizing section titles
\usepackage{tikz} % For drawing
\usepackage{braket} % For braket notation

\geometry{a4paper, margin=1in}

% Header and Footer
\pagestyle{fancy}
\fancyhf{}
\fancyhead[L]{Chapter 4 - Solutions to Try it}
\fancyhead[R]{\thepage}
\fancyfoot[C]{\thepage}

% Title Formatting
\titleformat{\section}{\Large\bfseries}{\thesection}{1em}{}
\titleformat{\subsection}{\large\bfseries}{\thesubsection}{1em}{}

% Title Page
\title{\textbf{Chapter 4} \\ \small Solutions to Try it}
\author{
    MohamadAli Khajeian\footnote{khajeian@ut.ac.ir} \\ 
    \small \textit{Faculty of Engineering Sciences, University of Tehran, Iran} \\ 
}
\date{\today}

\begin{document}

\maketitle

\begin{abstract}
    This document presents the solution of "Quantum Computing Explained by David McMAHON" exercises.
\end{abstract}

\section*{Try it - (page 76)}
The basis states for \( H \equiv \mathbb{C}^4 \) can be constructed by using \( |+\rangle, |-\rangle \) as the basis for \( H_1 \) and \( H_2 \).
\begin{equation*}
|w_1\rangle = |+\rangle|+\rangle
\end{equation*}
\begin{equation*}
|w_2\rangle = |+\rangle|-\rangle
\end{equation*}
\begin{equation*}
|w_3\rangle = |-\rangle|+\rangle 
\end{equation*}
\begin{equation*}
|w_4\rangle = |-\rangle|-\rangle
\end{equation*}
we have
\[
\langle w_3 | w_3 \rangle = (\langle -| \langle +|)(|- \rangle |+ \rangle) = \langle -|- \rangle \langle +|+ \rangle = (1)(1) = 1
\]
\[
\langle w_4 | w_4 \rangle = (\langle -| \langle -|)(|- \rangle |-\rangle) = \langle -|- \rangle \langle -|-\rangle = (1)(1) = 1
\]
\[
\langle w_2 | w_3 \rangle = (\langle +| \langle -|)(|- \rangle |+\rangle) = \langle +|- \rangle \langle -|+\rangle = (0)(0) = 0
\]
\[
\langle w_3 | w_2 \rangle = (\langle -| \langle +|)(|+ \rangle |- \rangle) = \langle -|+ \rangle \langle +|- \rangle = (0)(0) = 0
\]
\section*{Try it - (page 77)}
Given that \( \langle a | b \rangle = 1 \) and \( \langle c | d \rangle = -2 \), calculate \( \langle \psi | \phi \rangle \), where
\[
|\psi\rangle = |a\rangle \otimes |c\rangle \quad \text{and} \quad |\phi\rangle = |b\rangle \otimes |d\rangle.
\]
we have
\[
\braket{\psi | \phi} = (\bra{a} \bra{c})(\ket{b} \ket{d}) = \braket{a | b} \braket{c | d} = (1)(-2) = -2
\]
\section*{Try it - (page 77)}
Yes it can. Let
\[
|\phi\rangle = \frac{|0\rangle + |1\rangle}{\sqrt{2}} \quad \text{and} \quad |\chi\rangle = \frac{|0\rangle + |1\rangle}{\sqrt{2}}.
\]
then
\[
|\psi\rangle = |\phi\rangle \otimes |\chi\rangle = \left( \frac{|0\rangle + |1\rangle}{\sqrt{2}} \right) \otimes \left( \frac{|0\rangle + |1\rangle}{\sqrt{2}} \right) = \frac{1}{2} (|0\rangle |0\rangle + |0\rangle |1\rangle + |1\rangle |0\rangle + |1\rangle |1\rangle).
\]
\section*{Try it - (page 78)}
To calculate the tensor product of
\[
|a\rangle = \frac{1}{\sqrt{2}} \begin{pmatrix} 1 \\ 1 \end{pmatrix} \quad \text{and} \quad |b\rangle = \begin{pmatrix} 2 \\ 3 \end{pmatrix}
\]
we have
\[
|a\rangle \otimes |b\rangle = \frac{1}{\sqrt{2}} \begin{pmatrix} 1 \\ 1 \end{pmatrix} \otimes \begin{pmatrix} 2 \\ 3 \end{pmatrix} = \frac{1}{\sqrt{2}} \begin{pmatrix} 1 \\ 1 \end{pmatrix} \otimes \begin{pmatrix} 2 \\ 3 \end{pmatrix}
\]
then
\[
|a\rangle \otimes |b\rangle = \frac{1}{\sqrt{2}} \begin{pmatrix} 2 \\ 3 \\ 2 \\ 3 \end{pmatrix}
\]
\section*{Try it - (page 79)}
Given that \( X|0\rangle = |1\rangle \) and \( Z|{1}\rangle = -|1\rangle \), to calculate \( X \otimes Z |\psi\rangle \) where \( |\psi\rangle = |0\rangle \otimes |1\rangle \), we have
\[
X \otimes Z |\psi\rangle = (X \otimes Z)(|0\rangle \otimes |1\rangle)
\]
then we distribute the operators
\[
X \otimes Z |\psi\rangle = (X \otimes Z)(|0\rangle \otimes |1\rangle) = X|0\rangle \otimes Z|1\rangle
\]
next we use  \( X|0\rangle = |1\rangle \) and \( Z|{1}\rangle = -|1\rangle \) to write
\[
X|0\rangle \otimes Z|1\rangle = |1\rangle \otimes -|1\rangle
\]
since \( |\phi\rangle \otimes (\alpha|\chi\rangle) = \alpha |\phi\rangle \otimes |\chi\rangle \), so we can pull the scalars to the outside
\[
|1\rangle \otimes -|1\rangle = -(|1\rangle \otimes |1\rangle)
\]
we have shown that
\[
X \otimes Z |\psi\rangle = -(|1\rangle \otimes |1\rangle)
\]
\section*{Try it - (page 82)}
\( A \) and \( B \) are unitary matrices. therefore we have
\begin{equation}
   \label{uni}
   A A^\dagger = A^\dagger A = I, \quad \quad B B^\dagger = B^\dagger B = I.   
\end{equation}
to prove that \( A \otimes B \) is unitary, we need to show that \( (A \otimes B) (A \otimes B)^\dagger = (A \otimes B)^\dagger (A \otimes B) = I \) by considering its action on arbitrary vectors in the tensor product space. 
assume
\begin{equation*}
\ket{\psi} = \ket{\alpha} \otimes \ket{\beta}
\end{equation*}
\begin{equation*}
\ket{\phi} = \ket{\mu} \otimes \ket{v}
\end{equation*}
let \( C = (A \otimes B) \) we need to show
\begin{equation*}
   \braket{\psi | C C^\dagger | \phi} = \braket{\psi | C^\dagger C | \phi} = \braket{\psi | I | \phi} = \braket{\psi | \phi}.
\end{equation*}
we know that
\[
(A \otimes B)(|v\rangle \otimes |w\rangle) = (A|v\rangle) \otimes (B|w\rangle).
\]
by the definition of the tensor product and \( (A \otimes B)^\dagger = A^\dagger \otimes B^\dagger \), we can compute this as follows
\begin{align*}
   \braket{\psi | C^\dagger C | \phi} &= \langle \alpha | \otimes \langle \beta | (A \otimes B)^\dagger (A \otimes B) | \mu \rangle \otimes | v \rangle \\
   &= \langle \alpha | \otimes \langle \beta | ( A^\dagger A \otimes B^\dagger B) | \mu \rangle \otimes | v \rangle   \\
   &= \langle \alpha | A^\dagger A | \mu \rangle \cdot \langle \beta | B^\dagger B | v \rangle.   
\end{align*}
since \ref{uni},
\[
\langle \alpha | A^\dagger A | \mu \rangle = \langle \alpha | \mu \rangle, \quad \langle \beta | B^\dagger B | v \rangle = \langle \beta | v \rangle.
\]
thus,
\[
   \langle \alpha | \otimes \langle \beta | (A \otimes B)^\dagger (A \otimes B) | \mu \rangle \otimes | v \rangle = \langle \alpha | \mu \rangle \langle \beta | v \rangle  = \langle \alpha | \otimes \langle \beta | \mu \rangle \otimes | v \rangle.
\]
therefore, \( (A \otimes B)^\dagger (A \otimes B) = I \). we can also repeat it to prove \( (A \otimes B) (A \otimes B)^\dagger = I \),  which implies that \( A \otimes B \) is unitary.
\section*{Try it - (page 82)}
since \( Z\ket{0} = \ket{0}, Z\ket{1} = -\ket{1} \), we have
\begin{align*}
   Z \otimes I \ket{\psi} &= Z \otimes I \left( \frac{\ket{00} + \ket{11}}{\sqrt{2}} \right) \\
   &= \frac{1}{\sqrt{2}} \left[ (Z\ket{0})\ket{0} + (Z\ket{1})\ket{1} \right] \\
   &= \frac{\ket{00} - \ket{11}}{\sqrt{2}}
\end{align*}

\section*{Try it - (page 84)}
Let's write down the Pauli matrices
\[
X = \begin{pmatrix} 0 & 1 \\ 1 & 0 \end{pmatrix}, \quad Z = \begin{pmatrix} 1 & 0 \\ 0 & -1 \end{pmatrix}
\]
now we have
\[
X \otimes Z = 
\begin{pmatrix}
(0)Z & (1)Z \\
(1)Z & (0)Z
\end{pmatrix}
=
\begin{pmatrix}
0 & 0 & 1 & 0 \\
0 & 0 & 0 & -1 \\
1 & 0 & 0 & 0 \\
0 & -1 & 0 & 0
\end{pmatrix}
\]
and 
\[
Z \otimes X = 
\begin{pmatrix}
(1)X & (0)X \\
(0)X & (-1)X
\end{pmatrix}
=
\begin{pmatrix}
0 & 1 & 0 & 0 \\
1 & 0 & 0 & 0 \\
0 & 0 & 0 & -1 \\
0 & 0 & -1 & 0
\end{pmatrix}
\]
we can see that \(Z \otimes X \neg X \otimes Z\).
\end{document}
