\documentclass{article}
\usepackage{amsmath}  % For advanced math typesetting
\usepackage{amsfonts} % For math fonts
\usepackage{amssymb}  % For additional symbols
\usepackage{graphicx} % For including images
\usepackage{geometry} % For page layout
\usepackage{fancyhdr} % For header and footer
\usepackage{setspace} % For line spacing
\usepackage{hyperref} % For hyperlinks
\usepackage{titlesec} % For customizing section titles
\usepackage{tikz} % For drawing
\usepackage{braket} % For braket notation

\geometry{a4paper, margin=1in}

% Header and Footer
\pagestyle{fancy}
\fancyhf{}
\fancyhead[L]{Chapter 5 - Solutions to Try it}
\fancyhead[R]{\thepage}
\fancyfoot[C]{\thepage}

% Title Formatting
\titleformat{\section}{\Large\bfseries}{\thesection}{1em}{}
\titleformat{\subsection}{\large\bfseries}{\thesubsection}{1em}{}

% Title Page
\title{\textbf{Chapter 5} \\ \small Solutions to Try it}
\author{
    MohamadAli Khajeian\footnote{khajeian@ut.ac.ir} \\ 
    \small \textit{Faculty of Engineering Sciences, University of Tehran, Iran} \\ 
}
\date{\today}

% Commands
\newcommand{\op}[2]{|#1\rangle \langle#2|}
\newcommand{\sand}[3]{\braket{#1 | #2 | #3}}
\newcommand{\sandop}[3]{\braket{#1 #2 #3}}

\begin{document}

\maketitle

\begin{abstract}
    This document presents the solution of "Quantum Computing Explained by David McMAHON" exercises.
\end{abstract}

\section*{Try it - (page 89)}
\begin{equation*}
   \ket{\psi} = \frac{1}{2}\ket{u_1} + \frac{1}{\sqrt{2}}\ket{u_2} + \frac{1}{2}\ket{u_3}
\end{equation*}
then, we have
\begin{align*}
   \rho &= \op{\psi}{\psi} \\ 
   &= (\frac{1}{2}\ket{u_1} + \frac{1}{\sqrt{2}}\ket{u_2} + \frac{1}{2}\ket{u_3})(\frac{1}{2}\bra{u_1} + \frac{1}{\sqrt{2}}\bra{u_2} + \frac{1}{2}\bra{u_3}) \\
   &= \frac{1}{4}\op{u_1}{u_1} + \frac{1}{2\sqrt{2}}\op{u_1}{u_2} + \frac{1}{4}\op{u_1}{u_3} + \frac{1}{2\sqrt{2}}\op{u_2}{u_1} + \frac{1}{2}\op{u_2}{u_2}
   + \frac{1}{2\sqrt{2}}\op{u_2}{u_3} \\
   &+ \frac{1}{4}\op{u_3}{u_1} + \frac{1}{2\sqrt{2}}\op{u_3}{u_2} + \frac{1}{4}\op{u_3}{u_3}
\end{align*}
the trace is
\begin{align*}
   \text{Tr}(\rho) &= \sum_{i=1}^{3} \sand{u_i}{\rho}{u_i} = \sand{u_1}{\rho}{u_1} + \sand{u_2}{\rho}{u_2} + \sand{u_3}{\rho}{u_3} \\
   &= \sum_{i=1}^{3} \frac{1}{4}\sandop{u_i}{\op{u_1}{u_1}}{u_i} + \frac{1}{2\sqrt{2}}\sandop{u_i}{\op{u_1}{u_2}}{u_i} + \frac{1}{4}\sandop{u_i}{\op{u_1}{u_3}}{u_i} + \frac{1}{2\sqrt{2}}\sandop{u_i}{\op{u_2}{u_1}}{u_i} \\
   &+ \frac{1}{2}\sandop{u_i}{\op{u_2}{u_2}}{u_i} + \frac{1}{2\sqrt{2}}\sandop{u_i}{\op{u_2}{u_3}}{u_i} + \frac{1}{4}\sandop{u_i}{\op{u_3}{u_1}}{u_i} + \frac{1}{2\sqrt{2}}\sandop{u_i}{\op{u_3}{u_2}}{u_i} \\
   &+ \frac{1}{4}\sandop{u_i}{\op{u_3}{u_3}}{u_i} \\
   &= \frac{1}{4} + \frac{1}{2} + \frac{1}{4} = 1
\end{align*}
\section*{Try it - (page 96)}
No. because we have
\begin{equation*}
   \rho = \op{0}{0} + \op{1}{1}
\end{equation*}
then, the trace is
\begin{equation*}
   \text{Tr}(\rho) = 1 + 1 = 2 \neq 1.
\end{equation*}
\section*{Try it - (page 98)}
\begin{equation*}
   \ket{\psi} = \frac{2}{3}\ket{0} + \frac{\sqrt{5}}{3}\ket{1}
\end{equation*}
to show the state is normalized
\begin{equation*}
   \braket{\psi|\psi} = (\frac{2}{3}\bra{0} + \frac{\sqrt{5}}{3}\bra{1})(\frac{2}{3}\ket{0} + \frac{\sqrt{5}}{3}\ket{1}) = \frac{4}{9} + \frac{5}{9} = 1
\end{equation*}
and to get density matrix, we have
\begin{align*}
   \rho &= \op{\psi}{\psi} = (\frac{2}{3}\ket{0} + \frac{\sqrt{5}}{3}\ket{1})(\frac{2}{3}\bra{0} + \frac{\sqrt{5}}{3}\bra{1}) \\
   &= \frac{4}{9}\op{0}{0} + \frac{2\sqrt{5}}{9}\op{0}{1} + \frac{2\sqrt{5}}{9}\op{1}{0} + \frac{5}{9}\op{1}{1}
\end{align*}
then 
\[
\rho = 
\begin{pmatrix}
   \sand{0}{\rho}{0} & \sand{0}{\rho}{1} \\[0.6em]
   \sand{1}{\rho}{0} & \sand{1}{\rho}{1} 
\end{pmatrix}
=
\begin{pmatrix}
   \displaystyle\frac{4}{9} & \displaystyle\frac{2\sqrt{5}}{9} \\[1em]
   \displaystyle\frac{2\sqrt{5}}{9} & \displaystyle\frac{5}{9}
\end{pmatrix}.
\]
\section*{Try it - (page 99)}
According to Example 5.4,
\begin{equation*}
   \rho =  
   \begin{pmatrix}
      \displaystyle\frac{1}{5} & \displaystyle\frac{2}{5} \\[1em]
      \displaystyle\frac{2}{5} & \displaystyle\frac{4}{5}
   \end{pmatrix}
\end{equation*}
let's do the multiplication
\begin{align*}
   \rho \text{Z} &= \big(\frac{1}{5}\op{0}{0} + \frac{2}{5}\op{0}{1} + \frac{2}{5}\op{1}{0} + \frac{4}{5}\op{1}{1}\big)\big(\op{0}{0}-\op{1}{1}\big) \\
   &= \frac{1}{5}\op{0}{0} - \frac{2}{5}\op{0}{1} + \frac{2}{5}\op{1}{0} - \frac{4}{5}\op{1}{1}
\end{align*}
so
\begin{equation*}
   \braket{\text{Z}} = \text{Tr}(\rho \text{Z}) = \sum_{i=0}^{1} \sand{i}{\rho \text{Z}}{i}  = \frac{1}{5} - \frac{4}{5} = -\frac{3}{5}.
\end{equation*}
\section*{Try it - (page 99)}
\subsection*{(a)}
\begin{align}
   \text{P}_{-} &= \op{-}{-} = \big(\frac{1}{\sqrt{2}}(\ket{0} - \ket{1})\big)\big(\frac{1}{\sqrt{2}}(\bra{0} - \bra{1})\big) \nonumber \\
   &= \frac{1}{2}\big(\op{0}{0} - \op{0}{1} - \op{1}{0} + \op{1}{1}\big) \nonumber \\
   &= \frac{1}{2}
   \begin{pmatrix}
   1 & -1 \\[0.4em]
   -1 & 1   
   \end{pmatrix}. \label{eq1}
\end{align}
\subsection*{(b)}
According to Example 5.4,
\begin{equation*}
   \rho =  
   \begin{pmatrix}
      \displaystyle\frac{1}{5} & \displaystyle\frac{2}{5} \\[1em]
      \displaystyle\frac{2}{5} & \displaystyle\frac{4}{5}
   \end{pmatrix}
\end{equation*}
using \ref{eq1},
\begin{align*}
   \rho \text{\text{P}}_{-} &= \big(\frac{1}{5}\op{0}{0} + \frac{2}{5}\op{0}{1} + \frac{2}{5}\op{1}{0} + \frac{4}{5}\op{1}{1}\big)\big(\frac{1}{2}\big(\op{0}{0} - \op{0}{1} - \op{1}{0} + \op{1}{1}\big)\big)\\
   &= \frac{1}{2}\bigg(\frac{1}{5}\op{0}{0} - \frac{1}{5}\op{0}{1} - \frac{2}{5}\op{0}{0} + \frac{2}{5}\op{0}{1} + \frac{2}{5}\op{1}{0} - \frac{2}{5}\op{1}{1} - \frac{4}{5}\op{1}{0} + \frac{4}{5}\op{1}{1}\bigg)
\end{align*}
to get probability that finding the system in $\ket{-}$
\begin{equation*}
   \text{Pr}(\ket{-}) = \text{Tr}(\rho \text{P}_{-}) = \sum_{i=0}^{1} \sand{i}{\rho \text{P}_{-}}{i} = \frac{1}{2}\bigg(\big(\frac{1}{5} - \frac{2}{5}\big) + (\frac{4}{5} - \frac{2}{5})\bigg) = \frac{1}{10}.
\end{equation*}
\section*{Try it - (page 103)}
\begin{equation*}
   \rho = \frac{3}{8}\op{+}{+} + \frac{5}{8}\op{-}{-}
\end{equation*}
to write it in $\{\ket{0}, \ket{1}\}$ basis
\begin{align*}
   \op{-}{-} &= \big(\frac{1}{\sqrt{2}}(\ket{0} - \ket{1})\big)\big(\frac{1}{\sqrt{2}}(\bra{0} - \bra{1})\big) \nonumber \\
   &= \frac{1}{2}\big(\op{0}{0} - \op{0}{1} - \op{1}{0} + \op{1}{1}\big) \nonumber\\
   \op{+}{+} &= \big(\frac{1}{\sqrt{2}}(\ket{0} + \ket{1})\big)\big(\frac{1}{\sqrt{2}}(\bra{0} + \bra{1})\big) \nonumber \\
   &= \frac{1}{2}\big(\op{0}{0} + \op{0}{1} + \op{1}{0} + \op{1}{1}\big) \nonumber
\end{align*}
then
\begin{align*}
   \rho &= \frac{3}{16}\big(\op{0}{0} + \op{0}{1} + \op{1}{0} + \op{1}{1}\big) \\
   &+ \frac{5}{16}\big(\op{0}{0} - \op{0}{1} - \op{1}{0} + \op{1}{1}\big) \\
   &= \frac{1}{2}\op{0}{0} - \frac{1}{8}\op{0}{1} - \frac{1}{8}\op{1}{0} + \frac{1}{2}\op{1}{1} 
\end{align*}
to get probability that finding the system in $\ket{1}$
\begin{align*}
   \rho \text{P}_{1} &= \big(\frac{1}{2}\op{0}{0} - \frac{1}{8}\op{0}{1} - \frac{1}{8}\op{1}{0} + \frac{1}{2}\op{1}{1}\big)\big(\op{1}{1}\big) \\
   &= \frac{1}{2}\op{1}{1} - \frac{1}{8}\op{0}{1}
\end{align*}
so
\begin{equation*}
   \text{Pr}(\ket{1}) = \text{Tr}(\rho \text{P}_{1}) = \sum_{i=0}^{1} \sand{i}{\rho \text{P}_{1}}{i} = \frac{1}{2}
\end{equation*}
Therefore the state of the system after measurement is
\begin{equation*}
   \rho \rightarrow \frac{\text{P}_{1} \rho \text{P}_{1}}{\text{Tr}(\rho \text{P}_{1})} = \frac{(\op{1}{1})(\frac{1}{2}\op{1}{1} - \frac{1}{8}\op{0}{1})}{(1/2)} = \frac{(1/2)(\op{1}{1})}{(1/2)} = \op{1}{1}.
\end{equation*}
\section*{Try it - (page 108)}
\begin{equation*}
   \rho =  
   \begin{pmatrix}
      \displaystyle\frac{3}{5} & \displaystyle\frac{1}{5} \\[1em]
      \displaystyle\frac{1}{5} & \displaystyle\frac{2}{5}
   \end{pmatrix}
\end{equation*}
so
\begin{equation*}
   \rho = \frac{3}{5}\op{0}{0} + \frac{1}{5}\op{0}{1} + \frac{1}{5}\op{1}{0} + \frac{2}{5}\op{1}{1}
\end{equation*}
we can check does it valid density operator \\
\textbf{(1)}
\begin{equation*}
   \rho^\dagger = \frac{3}{5}\op{0}{0} + \frac{1}{5}\op{0}{1} + \frac{1}{5}\op{1}{0} + \frac{2}{5}\op{1}{1} = \rho
\end{equation*}
\textbf{(2)}
\begin{equation*}
   \text{Tr}(\rho) = \sum_{i=0}^{1} \sand{i}{\rho}{i}  = \frac{3}{5} + \frac{2}{5} = 1
\end{equation*}
\textbf{(3)}
\begin{align*}
   \det |\rho - \lambda\text{I}| &= \bigg|\frac{3}{5}\op{0}{0} + \frac{1}{5}\op{0}{1} + \frac{1}{5}\op{1}{0} + \frac{2}{5}\op{1}{1} - \lambda\op{0}{0} - \lambda\op{1}{1}\bigg| \\
   &=  \bigg|(\frac{3}{5} - \lambda)\op{0}{0} + \frac{1}{5}\op{0}{1} + \frac{1}{5}\op{1}{0} + (\frac{2}{5} - \lambda)\op{1}{1}\bigg| \\
   &= (\frac{3}{5} - \lambda)(\frac{2}{5} - \lambda) - (\frac{1}{5})(\frac{1}{5}) \\
   &= \frac{6}{25} - \frac{3}{5}\lambda - \frac{2}{5}\lambda + \lambda^2 - \frac{1}{25} \\
   &=  \lambda^2 - \lambda + \frac{1}{5} \\
   &=  5\lambda^2 - 5\lambda + 1 = 0
\end{align*}
so we can get eigenvalues
\begin{align*}
   \lambda_{1,2} = \frac{5 \pm \sqrt{(-5)^2 - 4(5)(1)}}{10} = \frac{5 \pm \sqrt{5}}{10} \ge 0
\end{align*}
since we have non-negetive eigenvalues, $\rho$ is Hermitian and Trace of $\rho$ is 1, therefore this density operator is valid.
to show this is mixed state,
\begin{align*}
   \rho^2 &= \big(\frac{3}{5}\op{0}{0} + \frac{1}{5}\op{0}{1} + \frac{1}{5}\op{1}{0} + \frac{2}{5}\op{1}{1}\big)\big(\frac{3}{5}\op{0}{0} + \frac{1}{5}\op{0}{1} + \frac{1}{5}\op{1}{0} + \frac{2}{5}\op{1}{1}\big) \\
   &= \frac{9}{25}\op{0}{0} + \frac{3}{25}\op{0}{1} + \frac{1}{25}\op{0}{0} + \frac{2}{25}\op{0}{1} + \frac{3}{25}\op{1}{0} + \frac{1}{25}\op{1}{1} + \frac{2}{25}\op{1}{0} + \frac{4}{25}\op{1}{1} \\
\end{align*}
the trace of $\rho^2$ is
\begin{equation*}
   \text{Tr}(\rho^2) = \sum_{i=0}^{1} \sand{i}{\rho^2}{i} = \frac{9}{25} + \frac{1}{25} + \frac{1}{25} + \frac{4}{25} = \frac{15}{25} = \frac{3}{5} < 1
\end{equation*}
to calculate $\braket{\text{Z}}$
\begin{align*}
   \rho \text{Z} &= \big(\frac{3}{5}\op{0}{0} + \frac{1}{5}\op{0}{1} + \frac{1}{5}\op{1}{0} + \frac{2}{5}\op{1}{1}\big)\big(\op{0}{0}-\op{1}{1}\big) \\
   &= \frac{3}{5}\op{0}{0} + \frac{1}{5}\op{1}{0}  - \frac{1}{5}\op{0}{1} - \frac{2}{5}\op{1}{1}
\end{align*}
so
\begin{align*}
   \braket{\text{Z}} = \text{Tr}(\rho \text{Z}) = \sum_{i=0}^{1} \sand{i}{\rho \text{Z}}{i} = \frac{3}{5} - \frac{2}{5} = \frac{1}{5}.
\end{align*}
\section*{Try it - (page 113)}
\begin{align*}
   \ket{\beta_{10}} = \frac{\ket{0_A}\ket{0_B} - \ket{1_A}\ket{1_B}}{\sqrt{2}}
\end{align*}
we have
\begin{align*}
   \rho &= \op{\beta_{10}}{\beta_{10}} = (\frac{\ket{0_A}\ket{0_B} - \ket{1_A}\ket{1_B}}{\sqrt{2}})(\frac{\bra{0_A}\bra{0_B} - \bra{1_A}\bra{1_B}}{\sqrt{2}}) \\
   &= \frac{1}{2}\bigg(\ket{0_A}\ket{0_B}\bra{0_A}\bra{0_B} - \ket{0_A}\ket{0_B}\bra{1_A}\bra{1_B} - \ket{1_A}\ket{1_B}\bra{0_A}\bra{0_B} + \ket{1_A}\ket{1_B}\bra{1_A}\bra{1_B}\bigg)
\end{align*}
so the density operator for Alice is
\begin{align*}
   \rho_{A} = \text{Tr}_B(\rho) = \sum_{i=0}^{1} \sand{i_B}{\rho}{i_B} = \sand{0_B}{\rho}{0_B} + \sand{1_B}{\rho}{1_B} 
\end{align*}
we have
\begin{align*}
   \sand{0_B}{\rho}{0_B} &= \sand{0_B}{\frac{\ket{0_A}\ket{0_B}\bra{0_A}\bra{0_B} - \ket{0_A}\ket{0_B}\bra{1_A}\bra{1_B} - \ket{1_A}\ket{1_B}\bra{0_A}\bra{0_B} + \ket{1_A}\ket{1_B}\bra{1_A}\bra{1_B}}{2}}{0_B} \\
   &= \frac{1}{2}\bigg(\braket{0_B|0_B}\op{0_A}{0_A}\braket{0_B|0_B} - \braket{0_B|0_B}\op{0_A}{1_A}\braket{1_B|0_B} \\
   &- \braket{0_B|1_B}\op{1_A}{0_A}\braket{0_B|0_B} + \braket{0_B|1_B}\op{1_A}{1_A}\braket{1_B|0_B}\bigg) \\
   &= \frac{\op{0_A}{0_A}}{2}
\end{align*}
and
\begin{align*}
   \sand{1_B}{\rho}{1_B} &= \sand{1_B}{\frac{\ket{0_A}\ket{0_B}\bra{0_A}\bra{0_B} - \ket{0_A}\ket{0_B}\bra{1_A}\bra{1_B} - \ket{1_A}\ket{1_B}\bra{0_A}\bra{0_B} + \ket{1_A}\ket{1_B}\bra{1_A}\bra{1_B}}{2}}{1_B} \\
   &= \frac{1}{2}\bigg(\braket{1_B|0_B}\op{0_A}{0_A}\braket{0_B|1_B} - \braket{1_B|0_B}\op{0_A}{1_A}\braket{1_B|1_B} \\
   &- \braket{1_B|1_B}\op{1_A}{0_A}\braket{0_B|1_B} + \braket{1_B|1_B}\op{1_A}{1_A}\braket{1_B|1_B}\bigg) \\
   &= \frac{\op{1_A}{1_A}}{2}
\end{align*}
therefore
\begin{align*}
   \rho_{A} &= \text{Tr}_B(\rho) = \sum_{i=0}^{1} \sand{i_B}{\rho}{i_B} = \sand{0_B}{\rho}{0_B} + \sand{1_B}{\rho}{1_B} = \frac{\op{0_A}{0_A}+\op{1_A}{1_A}}{2} \\
   &= \frac{1}{2}\begin{pmatrix}1&0\\0&1\end{pmatrix}
\end{align*}
to show Alice have completely mixed state
\begin{align*}
   \rho_{A}^2 = \frac{\op{0_A}{0_A}+\op{1_A}{1_A}}{4}
\end{align*}
then we get trace
\begin{align*}
   \text{Tr}(\rho_{A}^2)= \frac{1}{4} + \frac{1}{4} = \frac{1}{2} < 1.
\end{align*}
\end{document}
