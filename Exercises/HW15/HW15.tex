\documentclass{article}
\usepackage{amsmath}  % For advanced math typesetting
\usepackage{amsfonts} % For math fonts
\usepackage{amssymb}  % For additional symbols
\usepackage{graphicx} % For including images
\usepackage{geometry} % For page layout
\usepackage{fancyhdr} % For header and footer
\usepackage{setspace} % For line spacing
\usepackage{hyperref} % For hyperlinks
\usepackage{titlesec} % For customizing section titles
\usepackage{tikz} % For drawing
\usepackage{braket} % For braket notation

\geometry{a4paper, margin=1in}

% Header and Footer
\pagestyle{fancy}
\fancyhf{}
\fancyhead[L]{Eigenvalues of Density Operators in Bloch Sphere}
\fancyhead[R]{\thepage}
\fancyfoot[C]{\thepage}

% Title Formatting
\titleformat{\section}{\Large\bfseries}{\thesection}{1em}{}
\titleformat{\subsection}{\large\bfseries}{\thesubsection}{1em}{}

% Title Page
\title{\textbf{EigenValues of Density Operator in Bloch Sphere}}
\author{
    MohamadAli Khajeian\footnote{khajeian@ut.ac.ir} \\ 
    \small \textit{Faculty of Engineering Sciences, University of Tehran, Iran} \\ 
}
\date{\today}

% Commands
\newcommand{\op}[2]{|#1\rangle \langle#2|}
\newcommand{\sand}[3]{\braket{#1 | #2 | #3}}
\newcommand{\sandop}[3]{\braket{#1 #2 #3}}

\begin{document}

\maketitle

\begin{abstract}
    This document presents the how to get EigenValues of Density Operator in Bloch Sphere.
\end{abstract}

\section*{Proof}
\begin{align*}
    \rho &= \frac{1}{2}\big(\text{I} + \vec{a}\vec{\sigma}\big) \\
    &= \frac{1}{2}\bigg(\text{I} + a_{x}\big(\op{0}{1}+\op{1}{0}\big) + a_{y}\big(-i\op{0}{1}+i\op{1}{0}\big) + a_{z}\big(\op{0}{0}-\op{1}{1}\big)\bigg)\\
    &= \frac{1}{2}\bigg((1+a_{z})\op{0}{0}+(a_{x}-ia_{y})\op{0}{1}+(a_{x}+ia_{y})\op{1}{0}+(1-a_{z})\op{1}{1}\bigg)
\end{align*}
where ${\displaystyle {\vec {a}}\in \mathbb{R} ^{3}} $ is called the Bloch vector. so we can get eigenvalues
\begin{align*}
    \det |\rho - \lambda\text{I}| &= \det \bigg|\big(\frac{(1+a_{z})}{2}-\lambda\big)\op{0}{0}+\frac{(a_{x}-ia_{y})}{2}\op{0}{1}+\frac{(a_{x}+ia_{y})}{2}\op{1}{0}+\big(\frac{(1-a_{z})}{2}-\lambda\big)\op{1}{1}\bigg|\\
    &= \big(\frac{(1+a_{z})}{2}-\lambda\big)\big(\frac{(1-a_{z})}{2}-\lambda\big) - \big(\frac{(a_{x}-ia_{y})}{2}\big)\big(\frac{(a_{x}+ia_{y})}{2}\big) \\
    &= \big(\frac{(1-a^2_{z})}{4} - \lambda + \lambda^2\big) - \frac{a^2_{x} + a^2_{y}}{4} \\
    &= \lambda^2 - \lambda + \frac{1-(a^2_{x}+a^2_{y}+a^2_{z})}{4}
\end{align*}
\begin{align*}
   \lambda_{1,2} = \frac{1 \pm \sqrt{(-1)^2 - 4(1)(\displaystyle\frac{1-(a^2_{x}+a^2_{y}+a^2_{z})}{4})}}{2} = \frac{1 \pm \sqrt{a^2_{x}+a^2_{y}+a^2_{z}}}{2} = \frac{1 \pm r}{2}
\end{align*}
where $r=\sqrt{a^2_{x}+a^2_{y}+a^2_{z}}$.
\end{document}
